\documentclass[a4paper,12pt,twoside]{memoir}

%otros
\usepackage{listings}
\interfootnotelinepenalty=10000
\usepackage{makecell}
\usepackage{listings}
\lstset{
  basicstyle=\ttfamily,
  columns=fullflexible,
  frame=single,
  breaklines=true,
  postbreak=\mbox{\textcolor{red}{$\hookrightarrow$}\space},
}
\usepackage{dirtree}

% Castellano
\usepackage[spanish,es-tabla]{babel}
\selectlanguage{spanish}
\usepackage[utf8]{inputenc}
\usepackage[T1]{fontenc}
\usepackage{lmodern} % Scalable font
\usepackage{microtype}
\usepackage{placeins}

\RequirePackage{booktabs}
\RequirePackage[table]{xcolor}
\RequirePackage{xtab}
\RequirePackage{multirow} 

% Links
\usepackage[colorlinks]{hyperref}
\hypersetup{
	allcolors = {blue}
}

% Ecuaciones
\usepackage{amsmath}
\usepackage{multirow}

% Rutas de fichero / paquete
\newcommand{\ruta}[1]{{\sffamily #1}}

% Párrafos
\nonzeroparskip

% Huérfanas y viudas
\widowpenalty100000
\clubpenalty100000

% Evitar solapes en el header
\nouppercaseheads

% Imagenes
\usepackage{graphicx}
\newcommand{\imagen}[4]{
	\begin{figure}[!h]
		\centering
		\includegraphics[width=0.9\textwidth]{#1}
		\caption[{#3}]{#2}\label{#4}
	\end{figure}
	\FloatBarrier
}

\newcommand{\imagenflotante}[2]{
	\begin{figure}%[!h]
		\centering
		\includegraphics[width=0.9\textwidth]{#1}
		\caption{#2}\label{fig:#1}
	\end{figure}
}



% El comando \figura nos permite insertar figuras comodamente, y utilizando
% siempre el mismo formato. Los parametros son:
% 1 -> Porcentaje del ancho de página que ocupará la figura (de 0 a 1)
% 2 --> Fichero de la imagen
% 3 --> Texto a pie de imagen
% 4 --> Etiqueta (label) para referencias
% 5 --> Opciones que queramos pasarle al \includegraphics
% 6 --> Opciones de posicionamiento a pasarle a \begin{figure}
\newcommand{\figuraConPosicion}[6]{%
  \setlength{\anchoFloat}{#1\textwidth}%
  \addtolength{\anchoFloat}{-4\fboxsep}%
  \setlength{\anchoFigura}{\anchoFloat}%
  \begin{figure}[#6]
    \begin{center}%
      \Ovalbox{%
        \begin{minipage}{\anchoFloat}%
          \begin{center}%
            \includegraphics[width=\anchoFigura,#5]{#2}%
            \caption{#3}%
            \label{#4}%
          \end{center}%
        \end{minipage}
      }%
    \end{center}%
  \end{figure}%
}

%
% Comando para incluir imágenes en formato apaisado (sin marco).
\newcommand{\figuraApaisadaSinMarco}[5]{%
  \begin{figure}%
    \begin{center}%
    \includegraphics[angle=90,height=#1\textheight,#5]{#2}%
    \caption{#3}%
    \label{#4}%
    \end{center}%
  \end{figure}%
}
% Para las tablas
\newcommand{\otoprule}{\midrule [\heavyrulewidth]}
%
% Nuevo comando para tablas pequeñas (menos de una página).
\newcommand{\tablaSmall}[5]{%
 \begin{table}[h!]
  \begin{center}
   \rowcolors {2}{gray!35}{}
   \begin{tabular}{#2}
    \toprule
    #4
    \otoprule
    #5
    \bottomrule
   \end{tabular}
   \caption{#1}
   \label{tabla:#3}
  \end{center}
 \end{table}
}

%
% Nuevo comando para tablas pequeñas (menos de una página).
\newcommand{\tablaSmallSinColores}[5]{%
 \begin{table}[h!]
  \begin{center}
   \begin{tabular}{#2}
    \toprule
    #4
    \otoprule
    #5
    \bottomrule
   \end{tabular}
   \caption{#1}
   \label{tabla:#3}
  \end{center}
 \end{table}
}

\newcommand{\tablaApaisadaSmall}[5]{%
\begin{landscape}
  \begin{table}
   \begin{center}
    \rowcolors {2}{gray!35}{}
    \begin{tabular}{#2}
     \toprule
     #4
     \otoprule
     #5
     \bottomrule
    \end{tabular}
    \caption{#1}
    \label{tabla:#3}
   \end{center}
  \end{table}
\end{landscape}
}

%
% Nuevo comando para tablas grandes con cabecera y filas alternas coloreadas en gris.
\newcommand{\tabla}[6]{%
  \begin{center}
    \tablefirsthead{
      \toprule
      #5
      \otoprule
    }
    \tablehead{
      \multicolumn{#3}{l}{\small\sl continúa desde la página anterior}\\
      \toprule
      #5
      \otoprule
    }
    \tabletail{
      \hline
      \multicolumn{#3}{r}{\small\sl continúa en la página siguiente}\\
    }
    \tablelasttail{
      \hline
    }
    \bottomcaption{#1}
    \rowcolors {2}{gray!35}{}
    \begin{xtabular}{#2}
      #6
      \bottomrule
    \end{xtabular}
    \label{tabla:#4}
  \end{center}
}

%
% Nuevo comando para tablas grandes con cabecera.
\newcommand{\tablaSinColores}[6]{%
  \begin{center}
    \tablefirsthead{
      \toprule
      #5
      \otoprule
    }
    \tablehead{
      \multicolumn{#3}{l}{\small\sl continúa desde la página anterior}\\
      \toprule
      #5
      \otoprule
    }
    \tabletail{
      \hline
      \multicolumn{#3}{r}{\small\sl continúa en la página siguiente}\\
    }
    \tablelasttail{
      \hline
    }
    \bottomcaption{#1}
    \begin{xtabular}{#2}
      #6
      \bottomrule
    \end{xtabular}
    \label{tabla:#4}
  \end{center}
}

%
% Nuevo comando para tablas grandes sin cabecera.
\newcommand{\tablaSinCabecera}[5]{%
  \begin{center}
    \tablefirsthead{
      \toprule
    }
    \tablehead{
      \multicolumn{#3}{l}{\small\sl continúa desde la página anterior}\\
      \hline
    }
    \tabletail{
      \hline
      \multicolumn{#3}{r}{\small\sl continúa en la página siguiente}\\
    }
    \tablelasttail{
      \hline
    }
    \bottomcaption{#1}
  \begin{xtabular}{#2}
    #5
   \bottomrule
  \end{xtabular}
  \label{tabla:#4}
  \end{center}
}



\definecolor{cgoLight}{HTML}{EEEEEE}
\definecolor{cgoExtralight}{HTML}{FFFFFF}

%
% Nuevo comando para tablas grandes sin cabecera.
\newcommand{\tablaSinCabeceraConBandas}[5]{%
  \begin{center}
    \tablefirsthead{
      \toprule
    }
    \tablehead{
      \multicolumn{#3}{l}{\small\sl continúa desde la página anterior}\\
      \hline
    }
    \tabletail{
      \hline
      \multicolumn{#3}{r}{\small\sl continúa en la página siguiente}\\
    }
    \tablelasttail{
      \hline
    }
    \bottomcaption{#1}
    \rowcolors[]{1}{cgoExtralight}{cgoLight}

  \begin{xtabular}{#2}
    #5
   \bottomrule
  \end{xtabular}
  \label{tabla:#4}
  \end{center}
}


\graphicspath{ {./img/} }

% Capítulos
\chapterstyle{bianchi}
\newcommand{\capitulo}[2]{
	\setcounter{chapter}{#1}
	\setcounter{section}{0}
	\chapter*{#2}
	\addcontentsline{toc}{chapter}{#1. #2}
	\markboth{#2}{#2}
}

% Apéndices
\renewcommand{\appendixname}{Apéndice}
\renewcommand*\cftappendixname{\appendixname}

\newcommand{\apendice}[1]{
	%\renewcommand{\thechapter}{A}
	\chapter{#1}
}

\renewcommand*\cftappendixname{\appendixname\ }

% Formato de portada
\makeatletter
\usepackage{xcolor}
\newcommand{\tutor}[1]{\def\@tutor{#1}}
\newcommand{\course}[1]{\def\@course{#1}}
\definecolor{cpardoBox}{HTML}{E6E6FF}
\def\maketitle{
  \null
  \thispagestyle{empty}
  % Cabecera ----------------
\begin{center}%
	{\noindent\Huge Universidades de Burgos, León y Valladolid}\vspace{.5cm}%
	
	{\noindent\Large Máster universitario}\vspace{.5cm}%
	
	{\noindent\Huge \textbf{Inteligencia de Negocio y Big~Data en Entornos Seguros}}\vspace{.5cm}%
\end{center}%

\begin{center}%
	\includegraphics[height=3cm]{img/escudoUBU} \hspace{1cm}
	\includegraphics[height=3cm]{img/escudoUVA} \hspace{1cm}
	\includegraphics[height=3cm]{img/escudoULE} \vspace{1cm}%
\end{center}%

  \vfill
  % Título proyecto y escudo informática ----------------
  \colorbox{cpardoBox}{%
    \begin{minipage}{.9\textwidth}
      \vspace{.5cm}\Large
      \begin{center}
      \textbf{TFM del Máster Inteligencia de Negocio y Big Data en Entornos Seguros}\vspace{.6cm}\\
      \textbf{\LARGE\@title{}}
      \end{center}
      \vspace{.2cm}
    \end{minipage}

  }%
  \hfill
  \vfill
  % Datos de alumno, curso y tutores ------------------
  \begin{center}%
  {%
    \noindent\LARGE
    Presentado por \@author{}\\ 
    en Universidad de Burgos --- \@date{}\\
    Tutores: \@tutor{}\\
  }%
  \end{center}%
  \null
  \cleardoublepage
  }
\makeatother

\newcommand{\nombre}{Ignacio Ponsoda Llorens} %%% cambio de comando

% Datos de portada
\title{Clasificación de individuos a partir de imágenes oculares y redes neuronales pre-entrenadas.}
\author{\nombre}
\tutor{Dr. José Francisco Díez Pastor y Dr. Pedro Latorre Carmona}
\date{\today}

\begin{document}

\maketitle


\newpage\null\thispagestyle{empty}\newpage


%%%%%%%%%%%%%%%%%%%%%%%%%%%%%%%%%%%%%%%%%%%%%%%%%%%%%%%%%%%%%%%%%%%%%%%%%%%%%%%%%%%%%%%%
\thispagestyle{empty}


\noindent
\begin{center}%
	{\noindent\Huge Universidades de Burgos, León y Valladolid}\vspace{.5cm}%
	
\begin{center}%
	\includegraphics[height=3cm]{img/escudoUBU} \hspace{1cm}
	\includegraphics[height=3cm]{img/escudoUVA} \hspace{1cm}
	\includegraphics[height=3cm]{img/escudoULE} \vspace{1cm}%
\end{center}%

	{\noindent\Large \textbf{Máster universitario en Inteligencia de Negocio y Big~Data en Entornos Seguros}}\vspace{.5cm}%
\end{center}%



\noindent D. José Francisco Díez Pastor, profesor del departamento de Ingeniería Informática, área de Lenguajes y Sistemas Informáticos, y
\noindent D. Pedro Latorre Carmona, profesor del departamento de Ingeniería Informática, área de Lenguajes y Sistemas Informáticos.

\noindent Exponen:

\noindent Que el alumno  \nombre, con DNI 21698927Z, ha realizado el Trabajo final de Máster en Inteligencia de Negocio y Big Data en Entornos Seguros 
          titulado 'Clasificación de individuos a partir de imágenes oculares y redes neuronales pre-entrenadas'. 

\noindent Y que dicho trabajo ha sido realizado por el alumno bajo la dirección de los que suscriben, en virtud de lo cual se autoriza su presentación y defensa.

\begin{center} %\large
En Burgos, {\large \today}
\end{center}

\vfill\vfill\vfill

% Author and supervisor
\begin{minipage}{0.45\textwidth}
\begin{flushleft} %\large
Vº. Bº. del Tutor:\\[2cm]
D. José Francisco Díez Pastor
\end{flushleft}
\end{minipage}
\hfill
\begin{minipage}{0.45\textwidth}
\begin{flushleft} %\large
Vº. Bº. del tutor:\\[2cm]
D. Pedro Latorre Carmona
\end{flushleft}
\end{minipage}
\hfill

\vfill

% para casos con solo un tutor comentar lo anterior
% y descomentar lo siguiente
%Vº. Bº. del Tutor:\\[2cm]
%D. nombre tutor


\newpage\null\thispagestyle{empty}\newpage




\frontmatter

% Abstract en castellano
\renewcommand*\abstractname{Resumen}
\begin{abstract}
La utilización de la biometría para mejorar la seguridad, principalmente en lo referente al acceso de dispositivos electrónicos, es un recurso ampliamente empleado en la actualidad. El iris es uno de los elementos biométricos que mayores dificultades presentan para su suplantación, y por ello, su utilización en este campo ha atraído la atención de la comunidad científica estas últimas dos décadas. 

Las redes neuronales han demostrado ser útiles para extraer características del iris. Esta extracción se puede llevar a cabo no solo entrenando una red neuronal desde cero, sino también adaptando una ya pre-entrenada.
Respecto a la utilización de técnicas de adaptación de la red neuronal se plantean dos preguntas. ¿Es necesario aislar el iris para llevar a cabo la adaptación de la red neuronal?¿Se puede aplicar también para conjuntos de datos pequeños?

En este proyecto se han adaptado redes neuronales, inicialmente entrenadas para clasificar diversos objetos, para que sean capaces de identificar a un individuo utilizando su imagen ocular. 

Para ello, se han utilizado dos enfoques. En un primer enfoque, las redes neuronales se han adaptado utilizando imágenes oculares completas, mientras que, para el segundo enfoque, se ha adaptado la red neuronal con imágenes oculares donde previamente se ha aislado el iris, ya que a priori, el iris la zona de la imagen ocular que mejor permite la identificación de individuos.

Además, se han utilizado técnicas de ampliación del \textit{dataset} original, a fin de contar con un mayor número de muestras de cada individuo y también, mejorar la robustez de las redes neuronales adaptadas.

Los resultados muestran que las mejores tasas de clasificación se han dado en el enfoque donde se utilizaba la imagen ocular completa, sin que las técnicas de ampliación del \textit{dataset} hayan permitido mejorar la tasa de clasificación. 

Como futuras líneas de trabajo, se establecen la utilización de redes neuronales pre-entrenadas distintas, así como testear el modelo con imágenes realizadas fuera del entorno académico.


\end{abstract}

\renewcommand*\abstractname{Descriptores}
\begin{abstract}
biometría, iris, redes neuronales
\end{abstract}

\clearpage

% Abstract en inglés 
\renewcommand*\abstractname{Abstract}
\begin{abstract}

  The use of biometrics to improve security, mainly in relation to access to electronic devices, is a widely used resource today. The iris is one of the biometric elements that present the greatest difficulties for its impersonation, and that is why its use in this field has attracted the attention of the scientific community in the last two decades.

  In this project, neural networks, initially trained to classify various objects, have been adapted to be able to identify an individual using their eye image.

  To do this, two approaches have been used. In the first approach, the neural networks have been adapted using complete eye images, while, for the second approach, the same has been done, but the iris area has been isolated, which is a priori the area of the eye image that best allows the identification of individuals.

  In addition, extension techniques of the original dataset have been used, in order to have a greater number of samples of each individual and also to improve the robustness of the adapted neural networks.

  The results show that the best classification rates have occurred in the approach where the complete ocular image was used, without improvement in the classification rate by the datasets that had used data augmentation techniques.

  In future lines of work, the use of different pre-trained neural networks is established, as well as testing the model with images taken outside the academic environment.

  \end{abstract}

\renewcommand*\abstractname{Keywords}
\begin{abstract}
biometrics, iris, neural networks, fine-tuning, data augmentation
\end{abstract}

\clearpage

% Indices
\tableofcontents

\clearpage

\listoffigures

\clearpage

\listoftables
\clearpage

\mainmatter

\part*{Memoria}
\addcontentsline{toc}{part}{Memoria}

\capitulo{1}{Introducción}

El iris se utiliza como elemento de reconocimiento biométrico, tanto por su inmutabilidad a lo largo del tiempo como por resultar un valor único y 
personal, que supone que dos personas no tendrían el mismo iris (05 Iris Recognition Developmen Techniques: A Comprehensive Review).

En este estudio, se compararán técnicas de redes neuronales completas como una combinación de redes neuronales para la segmentación del ojo, la extracción
de características y finalmente, con técnicas de Machine Learning.

Para ello, vamos a utilizar una red neuronal preentrenada de ImageNet a la que le aplicaremos fine-tuning y compararemos los resultados con la combinación de
una red neuronal preentrenada para la segmentación, un preprocesamiento para extraer y normalizar el iris, la red preentrenada ImageNet para extraer las características principales
(sin realizar fine-tuning) y machine learning para el cluster (identificación del individuo).

\capitulo{2}{Objetivos del proyecto}

Los objetivos del trabajo son principalmente tres:
\begin{itemize}
    \item Optimizar el código del TFG y hacerlo reproducible.
    \item Aplicar data augmentation para comprobar la robustez del modelo.
    \item Comparar la utilización de una única rede neuronal para reconocimiento de personas con la combinación de redes neuronales y machine learning.                                                                                                                                                                                                           
\end{itemize}
\capitulo{3}{Conceptos teóricos}

05 techniques review entre todos los elementos biometricos, como huellas, cara, iris, voz y , el iris recognition system (IRS) es el metodo con mayor eficiencia a la hora de determinar la identiidad
de las personas, ya que el iris es el mismo a lo largo de la vida de una persona y es úncio, incluso entre gemelos. Esto es incluso utilizado en los procesos 
post-mortem para poder determinar la pertenencia del cuerpo 06 Post-mortem iris recognition. El iris humanos es un organo del ojo, que tiene como funciona controlar el tamaño
de la pupila en función de la cantidad de luz que llega hasta este 06 Post-mortem iris recognition.

\section{Secciones}

Las secciones se incluyen con el comando section.

\subsection{Subsecciones}

Además de secciones tenemos subsecciones.

\subsubsection{Subsubsecciones}

Y subsecciones. 


\section{Referencias}

Las referencias se incluyen en el texto usando cite \cite{wiki:latex}. Para citar webs, artículos o libros \cite{koza92}.


\section{Imágenes}

Se pueden incluir imágenes con los comandos standard de \LaTeX, pero esta plantilla dispone de comandos propios como por ejemplo el siguiente:


\section{Listas de items}

Existen tres posibilidades:

\begin{itemize}
	\item primer item.
	\item segundo item.
\end{itemize}

\begin{enumerate}
	\item primer item.
	\item segundo item.
\end{enumerate}

\begin{description}
	\item[Primer item] más información sobre el primer item.
	\item[Segundo item] más información sobre el segundo item.
\end{description}
	
\begin{itemize}
\item 
\end{itemize}

\section{Tablas}

Igualmente se pueden usar los comandos específicos de \LaTeX o bien usar alguno de los comandos de la plantilla.

\tablaSmall{Herramientas y tecnologías utilizadas en cada parte del proyecto}{l c c c c}{herramientasportipodeuso}
{ \multicolumn{1}{l}{Herramientas} & App AngularJS & API REST & BD & Memoria \\}{ 
HTML5 & X & & &\\
CSS3 & X & & &\\
BOOTSTRAP & X & & &\\
JavaScript & X & & &\\
AngularJS & X & & &\\
Bower & X & & &\\
PHP & & X & &\\
Karma + Jasmine & X & & &\\
Slim framework & & X & &\\
Idiorm & & X & &\\
Composer & & X & &\\
JSON & X & X & &\\
PhpStorm & X & X & &\\
MySQL & & & X &\\
PhpMyAdmin & & & X &\\
Git + BitBucket & X & X & X & X\\
Mik\TeX{} & & & & X\\
\TeX{}Maker & & & & X\\
Astah & & & & X\\
Balsamiq Mockups & X & & &\\
VersionOne & X & X & X & X\\
} 

\capitulo{4}{Técnicas y herramientas}

Se ha utilizado el dataset de CASIA para la imágenes de ojos.

Así mismo, se ha utilizado la red neuronal preentrenada Basada en U-Net y accesible desde https://github.com/jus390/U-net-Iris-segmentation, la cual ya había sido
entrenada para la segmentación del iris.

Finalmente, para el último proceso del proyecto, se utiliza imagenet como red neuronal preentrenada central.

Además se ha utilizado python para todo el proyecto, esto quiere decir preprocesado, creación y utilización de redes neuronales y clasificación contécnicas de 
machine learning.

Entre las principales librerías utilizadas se encuentran:

* os, para el acceso a los directorios
* numpy, para trabajar con las imágenes a nivel de arrays
* scikit-image, para la transformación de las imágenes y el uso de dataset
* tensorflow, para la modificación de las redes neuronales
* keras, para el manejo de las redes neuronales
* matplot, para las gráficas



\capitulo{5}{Aspectos relevantes del desarrollo del proyecto}

El proyecto puede dividirse en una fase previa y tres propuestas diferentes para la identificación biomédica de personas a través de una imagen de su ojo.

La primera fase trata de la optimización del código previo y la creación de un pipeline con una configuración que permitiese un mejor control de las funciones. 
También del  data augnmentation, en el cual se aplica tanto ruido gausseano (de 2.5, 5 y 7.5) como transformaciones afines (), siendo realizadas de 
forma aleatoria, con lo cual los supuestos pueden ser a) imagen sin data augnmentation, b)imagen con ruido gausseano, c)imagen con transformaciones afines y 
d)imagen con ruido gausseano y transformaciones afines. Este dataset será el base para todo el proyecto.

La segunda fase se trata de la elección de la mejor forma de clasificar las imagenes de entrada:

1.La propuesta en el TFG anterior, en la cual se realiza un preprocesamiento de las imágenes del dataset original. Este preprocesamiento consiste primer lugar 
en la segmentación de las imágenes del iris con una red neuronal preentrenada precisamente para realizar esta acción. En segundo lugar, se realiza una extracción
del iris a través de una binarización de las partes del ojo y una extracción del iris, a la que se le aplica una normalización para que quede proyectado.
La siguiente fase de esta primera propuesta es la extracción de características (quitandole las dos últimas capas a una red neuronal) de la imagen normalizada
 para posteriormente utilizar una red neuronal preentrenada con imagenet (de hecho 3 redes de la cual se elige la mejor). Posteriormente, estas características
 extraidas se pasan a modelos de ML, que son los que realizarán la clasificación. 

 Todo el proyecto queda establecido en una pipeline

\capitulo{6}{Trabajos relacionados}

El principal apartado anterior lo podemos encontrar en el TFG de extracción del iris, en el cual se basa este trarbajo, puesto se realiza a grandes rasgos
todo lo relativo a la primera de las opciones del trabajo.

Sobre temas de extracción del iris encontramos 02 iris wavelet neural, donde se hace un preprocesamiento con extración del iris utilizando Hough Transform y la 
normalización con Daugmands rubber. Luego, tras eliminar el ruido, la extración se realiza con transformaciones de wavelet. Finalmente, se crea una red neutornal
utilizando el mean-squared error para calcular los pesos en la red.

En 03 deep iris encontramos el desarrollo de técnicas de deep learning para el reconocimiento del iris basado en convolutional neural network residual. utilizando una red preentrenada 
de ResNet50 y fine-tuning, entrenado con una cross-entropy loss function (but they are not using data augnmentation, they are using another dataset IIT Delhi dataset and they are 
not doing the p`reprocessing step).

In 13 ImageNet Deep CNN they use the ImageNet dataset with data augnmentation, dropout to train a nerual network to detect images of the feed (maybe we can remove it as
it is not related with iris).

En 14 experimental deep convolutional iris recognition, utilizaron el dataset CASIA - 10000 y la arquitectura VGG-Net, lo cual realiza un PCA para extraer los elementos
más característicos de las imágenes . Después utilizan algoritmos de clasificación para clasificar las imágenes, como el SVM (esto es similar al TFG) y consiguen unos percentajes
de reconocimiento muy altos.

05 techniques review habla de siete pasos en los que se divide un sistema de reconocimiento del iris, 1)adquisición, 2)preprocesamiento, 3)segmentación
 4)normalización, 5)extracción de características, 6) selección de features únicos y característicos, 7)clasificación. Este paper también describe una falta de
 trabajos entorno a datasets de baja cualidad (revisar para sacar más papers). El paper también realza que los sistemas de reconocimiento del iris (IRS) se vuelven 
 poco efectivos cuando las imagenes tienen rotaciones or reflejos, algo que intentamos de mejorar en nuestro proceso, añadiendo ruido con el data augnmentation.

 Este mismo paper también comenta los distintos dataset utilizados para estos estudios de reconocimiento de iris, el tipo de ruido utilizado así como su método,
 los tipos de segmentación tradicional y actual utilizados (habitualmente con redes neuronales), técnicas de normalización y extracciñon de caracterñisticas, así 
 como los tipos de accuracy de los métodos de iris recognition.
.... (utilizando Machine learning y deep learning)

Y para el tema de clasificación de personas con el ojo, encontramos ...

Por otro lado, en temas de fine tunning con redes neuronales, tenemos...
\capitulo{7}{Conclusiones y Líneas de trabajo futuras}

Explicación de la sección

\section{Conclusiones}
Como conclusion de este proyecto, la utilización de técnicas de machine learning, contando con ordenadores de pocos recursos, demuestra que el preprocesamiento
es necesario para centrar los procesos en los elementos de las imágenes verdaderamente importantes. Por otro lado, una vez que la imagen está preprocesada, la utilización 
de redes neuronales con fine tunning para clasificar las imágenees se ha resuelto como un mayor accuracy que el modelo que aplica machine learning en la última fase
pero este último ha demostrado ser más rápido, por lo tanto, la utilización de una u otra técnica variará según los recursos que se tengan y el contexto en el que se 
vaya a utilizar (inmediatex con que se necesiten los resultados).

\section{Líneas de trabajo futuras}

Las líneas de trabajo futuras se podrían determinan con la utilización de estas técnicas con nuevos datasets, consiiguiendo un modelo lo suficientemente robusto
que permitiese su utilización en un programa de escritorio, con una primera fase de fine tunning con imágenes del ojo del usuario y una segunda fase donde este 
se utilizase como método de seguridad para el acceso a ciertos docuementos de los aparatos electrónicos.


%\renewcommand\chaptername{Anexo}
%\renewcommand\thechapter{\Roman{chapter}}
%\setcounter{chapter}{0}

% Añadir entrada en el índice: Anexos
\appendix
\addcontentsline{toc}{part}{Apéndices}
\part*{Apéndices}

\apendice{Plan de Proyecto Software}

\section{Introducción}

La planificación del proyecto ha sido una parte esencial, puesto que ha permitido la coordinación entre estudiante y tutores de forma online.

\section{Planificación temporal}

Para la planificación temporal se ha utilizado la metodología SCRUM basada en sprints, generalmente de una o dos semanas.

\subsection{Sprint 1}

\begin{itemize}
\item Creación de un repositorio para el control de las versiones y el seguimiento del trabajo.
\item Investigación sobre la materia de estudio (antecedentes, casos de uso, estado del arte)
\item Investigación sobre posibles dataset de iris para el estudio.
\item Investigación sobre la metodogía que mejor se adapte al proyecto.
\item Estudio sobre el uso de data augmentation sobre datasets de iris.
\item Investigación sobre el uso de pipelines para la mejora de la reproducibilidad del trabajo.
\item Investigación sobre los transformes de scikit-learn
\end{itemize}

\subsection{Sprint 2}

\begin{itemize}
\item Investigación sobre pipelines que se ejecuten de forma condicional.
\item Investigación sobre posibles redes neuronales para la segmentación del iris.
\item Investigación para el uso de data augmentation en el paquete Keras.
\end{itemize}

\subsection{Sprint 3}

\begin{itemize}
\item Clarificación de objetivos del proyecto.
\item Pruebas con la utilización de ruido gaussiano con standard deviation aleatorio
\item Investigación sobre las transformaciones afines como método para el data augmentation.
\item Implementación del pipeline.
\end{itemize}

\subsection{Sprint 4}

\begin{itemize}
\item Pruebas con las transformaciones afines de Keras.
\end{itemize}

\subsection{Sprint 5}

\begin{itemize}
\item Pruebas con el transformador de Keras.
\end{itemize}

\subsection{Sprint 6 y 7}

\begin{itemize}
\item Aplicaciñon de fine tunning con redes neuronales.
\end{itemize}

\subsection{Sprint 8}

\begin{itemize}
\item Reconfigurar el pipeline para que el data augmentation solo se aplique al conjunto de entreamiento.
\item Prevención de imagenes no segmentadas correctamente para prevenir que entren en el proceso de normalización.
\item Incorporación del modelo de deep learning al pipeline.
\item Mejora de la configuración del pipeline.
\end{itemize}

\subsection{Sprint 9}

\begin{itemize}
\item Creación de los modelos de clasificación.
\end{itemize}

\section{Estudio de viabilidad}

En este apartado se redacta la viabilidad del proyecto, tanto económica como legal, que potencialmente habría costado el proyecto.

\subsection{Viabilidad económica}

En primer lugar, se encuentran los cálculos económicos del proyecto, que se pueden dividir en gastos de \textit{software}, \textit{hardware} y de personal.

\subsubsection{\textit{Software}}

Los recursos utilizados para la realización gratuitos y de código abierto, por lo que no se prevé ningún gasto en este apartado.

\subsubsection{\textit{Hardware}}

Para la realización del proyecto, se ha contado con una máquina de la Universidad de Burgos, cuyas características se reflejan en la tabla\ref{caracteristicaspc}.
En base a las características, el precio para esta máquina se establece entre los 2.500 y los 3.000 euros, aunque se ha de tener en cuenta que la amortización del mismo va más allá de la realización de este proyecto.

\subsubsection{Personal}

Finalmente, se ha de tener en cuenta que el proyecto se ha llevado a cabo por el estudiante y dos tutores.

Considerando que el proyecto se ha llevado a cabo en aproximadamente en 6 meses, se considerará que el estudiante trabajaría a tiempo completo durante ese tiempo,
con un sueldo de 1500 euros al mes, cuyo gasto total para la universidad sería de unos 2400 euros al mes. Tabla \ref{tabla:gastosestudiante}

\tablaSmall{Gastos de contratación del estudiante.}{l c}{gastosestudiante}
{ {Gasto} & Cantidad en euros\\}{ 
Base de cotización & 1500  \\
Coste seguridad social empresa & 898\\
Total universidad & 2398\\
} 


Por su parte, los dos profesores estarían contratados por ese mismo periodo de tiempo, pero trabajando a media jornada., con un salario de 1800 euros al mes por la media jornada, cuyo gasto para la unviersidad sería de unos 2753 euros al mes por cada tutor. Tabla \ref{tabla:gastostutores}

\tablaSmall{Gastos de contratación de los tutores (por tutor).}{l c}{gastostutores}
{ {Gasto} & Cantidad en euros\\}{ 
Base de cotización & 1800  \\
Coste seguridad social empresa & 953\\
Total universidad & 2753\\
} 
\subsection{Gasto total}

Por lo tanto, el gasto total se establece en, como se desglosa en la siguiente tabla \ref{tabla:gastostotales}

\tablaSmall{Gastos totales del proyecto.}{l c}{gastostotales}
{ {Gasto} & Cantidad en euros\\}{ 
\textit{Hardware} & 2500 \\
Contratación estudiante & 14400  \\
Contratación tutores & 33036\\
Total & 49936\\
} 

\subsection{Viabilidad legal}

Como se ha comentado anteriormente, el \textit{software} utilizado es gratuito y libre, y está sujeto a las licencias recogidas en la tabla \ref{tabla:licencialibrerias}.

\tablaSmall{Licencias de las librerías utilizadas.}{l c c c}{licencialibrerias}
{ {Librería} & Versión & Licencia\\}{ 
Anaconda & 2.1.1 & Libre\\
Imageio & 2.19.3 & BSD\\
Jupyter Notebook & 4.11 & BSD\\
Keras & 2.7 & MIT\\
Matplotlib & 3.1 & BSD\\
Numpy & 1.19.5 & BSD\\
OpenCV & 4.6 & BSD\\
Pandas & 1.4.3 & BSD\\
Python & 3.10.5 & PFS \\
Scikit-Learn & 1.0 & BSD\\
TensorFlow & 2.9.1 & Apache\\
} 

El \textit{dataset} utilizado tiene fines de investigación, pero no se permite la distribución alterada ni la misma y su uso no referenciado.

Finalmente, se han utilizado los iconos de \url{https://www.flaticon.com/} para la elaboración de las gráficas. La licencia de los mismos no permiten comercializarlos
 o modificarlos.

\apendice{Especificación de Requisitos}

\section{Introducción}

En este apartado se especifican los objetivos y requisitos del proyecto.

\section{Objetivos generales}

Los objetivos generales son, el ser capaz de asimilar el trabajo de \cite{tfg_iris_2020}, modificar el código para hacerlo reutilizable y, finalmente, ser capaz de 
utilizar técnicas de \textit{fine-tuning} con el \textit{dataset} de imágenes sobre una red neuronal ya pre-entrenada.

\section{Catalogo de requisitos}

Los requisitos del proyecto son los siguientes.

\begin{enumerate}
    \item Optimizar del código previo.
    \item Crear los modelos para la clasificación de imágenes oculares.
        \begin{enumerate}
            \item Dividir el \textit{dataset} en entrenamiento y test.
            \item Aplicar de \textit{data augmentation}.
            \item Segmentar de las imágenes.
            \item Normalizar de las imágenes.
        \end{enumerate}
    \item Aplicar \textit{fine-tuning} a los distintos \textit{datasets}.
    \item Crear una configuración donde el usuario pueda modificar los parámetros y visualizar los distintos \textit{outputs}.
    \item Determinar qué proceso de \textit{fine-tuning} proporciona una tasa de acierto más alta.
  \end{enumerate}

\section{Especificación de requisitos}

\subsection{Optimizar el código previo}

Asimilar y optimizar el código previo para que sea funcional. Para llevar a cabo este apartado es necesario contar con el código y el \textit{dataset}, ambos accesibles desde \url{https://github.com/jaa0124/iris_classifier/}.

\subsection{Crear los modelos para la clasificación de imágenes oculares}

La clasificación de las imágenes oculares se lleva a cabo utilizando modelos adaptados utilizando distintos \textit{datasets}. Su creación se lleva a cabo en las siguientes fases:

\subsubsection{Dividir el \textit{dataset} en entrenamiento y test}

División del \textit{dataset} de CASIA en entrenamiento y test, para poder validar los modelos. Los requisitos previos son el 1.

\subsubsection{Aplicar de \textit{data augmentation}}

Aplicación de \textit{data augmentation} sobre dos de los modelos. Los requisitos previos son el 1 y el 2a. Este paso solo se aplica a dos de los \textit{datasets}.

\subsubsection{Segmentar de las imágenes}

Segmentación de las imágenes donde el iris será aislado. Los requisitos previos son 1, 2a y 2b. Este paso solo se aplica a dos de los \textit{datasets}.

\subsubsection{Normalizar de las imágenes}

Normalización de las imágenes donde el iris será aislado. Los requisitos previos son 1, 2a, 2b y 2c. Este paso solo se aplica a dos de los \textit{datasets}.

\subsection{Aplicar \textit{fine-tuning} a los distintos \textit{datasets}}

Aplicación de \textit{fine-tuning} sobre los datasets. Los requisitos previos son  Los requisitos previos son 1, 2a, 2b, 2c y 2d. Dependiendo del \textit{dataset}, alguno de los pasos previos puede no ser necesario.

\subsection{Crear una configuración donde el usuario pueda modificar los parámetros y visualizar los distintos \textit{outputs}}

Creación de una \textit{pipeline} y establecimiento de su configuración para permitir al usuario adaptar el código a sus necesidades. Todos los pasos anteriores son requisitos previos.

\subsection{Determinar qué proceso de \textit{fine-tuning} proporciona una tasa de acierto más alta}

Determinación de los valores más altos de la tasa de acierto en los distintos modelos.




\apendice{Especificación de diseño}

\section{Introducción}

A continuación, se presentan las características asociadas al diseño del proyecto, con el fin de entender mejor como se han construido las distintas partes del proyecto.

\section{Diseño de datos}

\subsection{conjunto de datos} \label{anx:dataset}

Tal como se ha explicado en la memoria, se ha utilizado el conjunto de datos CASIA-V1.

La estructura de datos es la siguiente, por persona y sesión. Figura \ref{img:casia-dir}.

\imagenPequenya{img/17_CASIA_dir.png}{Estructura de directorios de CASIA-V1.}{Estructura de directorios de CASIA-V1.}{img:casia-dir}

\subsection{Jupyter \textit{Notebook}}

El código del proyecto se ha reducido a dos documentos de Jupyter \textit{notebook}:


\begin{itemize}
    \item El primero es el \texttt{pipeline.ipynb}, en el cual se ha reducido todas las secciones del proyecto, encapsuladas en una \textit{pipeline}.
    \item El segundo es el \texttt{accuracy.ipynb}, donde se lleva a cabo el cálculo de la tasa de acierto de los modelos.
\end{itemize}


\section{Diseño procedimental}

\begin{enumerate}
    \item Carga de datos
    \item \textit{Data augmentation}
    \item Segmentación y clasificación
    \item \textit{Fine-tuning}
    \item Tasa de acierto
    \end{enumerate}

\section{Diseño arquitectónico}

Con la utilización de un \textit{pipeline} configurable, se ha pretendido que el código utilizado en este proyecto pueda ser re-utilizado y permita modificar su configuración de una manera más sencilla.

\apendice{Documentación técnica de programación}

\section{Introducción}

En este apartado se determinará la forma de proceder para poder replicar el proyecto.

\section{Estructura de directorios}

\begin{itemize}
    \item 06\_Models contiene los modelos finales
    \item img contiene las imágenes utilizadas en el propio repositorio
    \item memoria, incluye la memoria
    \begin{itemize}
        \item tex, contiene los archivos tex
        \item img, contiene las imágenes del proyecto
    \end{itemize}
        \item src, incluye el código del proyecto con distintas versiones
\end{itemize}

\section{Manual del programador}


Los \textit{notebooks} del proyecto se pueden explorar en cualquier plataforma compatible con Jupyter notebooks.

\subsection{\textit{Notebook} \texttt{9-pipelines\_v7.ipynb}} \label{anx:pipeline}

El {\textit{notebook} \texttt{9-pipelines\_v7.ipynb} contiene el código principal del proyecto. A continuación se detallan sus partes.

\subsubsection{Configuración}

La configuración de la \textit{pipeline}se muestra a continuación. En ella, se establece primero una configuración general, es decir, que es aplicable a
las distintas secciones. En ella se han configurado las rutas en las que se guardan los distintos \textit{datasets}. Si todo el código es ejecutado, solo el
\textit{'root-dir'} se mantendrá, y el resto se irán sobreescribiendo cuando se cree el propio \textit{dataset}. No obstante, la capacidad de establecer todas
las rutas en la configuración es lo que nos permite cambiar el orden de las secciones en la \textit{pipeline}. 

\begin{itemize}
    \item \textit{tratardDataset}
    \begin{itemize}
        \item \textit{general\_train\_size}, tamaño del \textit{dataset} que será utilizado para el entrenamiento [Valor del 0-1].
        \item \textit{show\_first}, mostrar primera imagen [boolean].
    \end{itemize}
    \item \textit{dataAugmentation}
    \begin{itemize}
        \item \textit{gaussianNoise}, aplicación del ruido gaussiano [boolean].
        \item \textit{stdGN}, valores de ruido gaussiano a aplicar en las imágenes [valores menores de 10].
        \item \textit{afinTransformation}, aplicación de transformaciones afines [boolean].
    \end{itemize}
    \item \textit{segmentation}
    \begin{itemize}
        \item \textit{redNeuronal}, nombre de la red neuronal a utilizar [extensión .h5].
        \item \textit{verImagenV1}, ver la imagen producida [boolean].
    \end{itemize}
    \item \textit{CNN\_classification}
    \begin{itemize}
        \item \textit{dataset\_dir}, directorio a utilizar ['raw',para la imagen ocular completa o 'normalizado', para la imagen del iris aislada].
        \item \textit{verImagenV1}, ver la imagen producida [boolean].
        \item \textit{CNN\_weights}, establecer de donde utilizar los pesos para la red neuronal [imagenet, None, o se le pasa un directorio con los pesos].
        \item \textit{train\_size}, tamaño del \textit{dataset} de entrenamiento [0-1].
        \item \textit{test\_size}, tamaño del \textit{dataset} de testeo [0-1].
        \item \textit{batch\_size}, tamaño del \textit{batch} [debe de ser menor del tamaño del \textit{dataset}].
        \item \textit{epochs1}, establecer el tamaño del \textit{epochs} para entrenar al primer modelo [10-100].
        \item \textit{plt\_accuracy1}, mostrar gráfica con el \textit{accuracy} del primer modelo [boolean].
        \item \textit{epochs2}, establecer el tamaño del \textit{epochs} para entrenar al segundo modelo [8-80].
        \item \textit{plt\_accuracy2}, mostrar gráfica con el \textit{accuracy} del segundo modelo[boolean].
        \item \textit{epochs3}, establecer el tamaño del \textit{epochs} para entrenar al tercer modelo [8-50].
        \item \textit{plt\_accuracy3}, mostrar gráfica con el \textit{accuracy} del tercer modelo[boolean].
        \item \textit{results\_array}, mostrar el \textit{array} de resultados [boolean].
        \item \textit{save\_model}, guardar el modelo [boolean].
        \item \textit{save\_model\_name}, nombre con el que guardar el modelo [texto].
    \end{itemize}
\end{itemize}


\begin{lstlisting}[language=Python] 

confi_dict = {

    'general':{
        'root_dir':r"/home/root_folder",
        'dataset_dir': "CASIA-IrisV1",
        'dataset_unif_dir': r"./CASIA-IrisV1_unif",
        'dataset_unif_dir_aug':r"./CASIA-IrisV1_unif_aug",
        'dataset_unif_segv2_edg_norm_dir' : r"./CASIA-IrisV1_unif_segv2_edg_norm",
        'dataset_unif_dir':r"./CASIA-IrisV1_unif_aug",
        'dataset_unif_segv2_edg_norm_dir' : r"./CASIA-IrisV1_unif_aug_segv2_edg_norm",
        'dataset_unif_dir': r"./CASIA-IrisV1_reservado"
    },

    '1_tratarDataset':{
        'general_train_size': 0.7,
        'show_first' : False
    },

    '1.1_dataAugmentation':{
        'gaussianNoise' : True,
        'stdGN': [2.5, 5, 7.5],
        'afinTransformation': True
    },

    '2.1_segmentation':{
        'redNeuronal' : "Iris_unet_d5.h5", 
        'verImagenV1' : False
    },
    
    # Dataset dir can be "normalizado" or "raw"

    '4_CNN_classification' :{
        'dataset_dir' : "normalizado",
        'CNN_weights' : "imagenet",
        'train_size' : 0.7, 
        'test_size' : 0.3, 
        'batch_size' : 10,
        'epochs1' : 50, 
        'plt_accuracy1' : True,
        'epochs2' : 40, 
        'plt_accuracy2' : True,
        'epochs3' : 25, 
        'plt_accuracy3' : True,
        'results_array' : True,
        'save_model' : False, 
        'save_model_name' : "models/normalizado_aug_modelv1"
    }

}
\end{lstlisting}

\subsubsection{Secciones}

Las secciones con las que cuenta el \textit{pipeline} son las siguientes:

\begin{enumerate}
    \item \textit{tratar\_dataset\_casia}
    \begin{itemize}
        \item Modificación de la configuración de archivos vista en el Anexo \ref{anx:dataset} a una configuración de directorio único.
    \end{itemize}
    \item \textit{data\_augmentation}
    \begin{itemize}
        \item Aplicación de transformaciones afines.
        \item Aplicación del ruido gaussiano.
    \end{itemize}
    \item \textit{segmentation}
    \begin{itemize}
        \item Genera las muestras que se le pasaran a la red pre-entrenada.
        \item Segmenta el iris.
    \end{itemize}
    \item \textit{normalization}
    \begin{itemize}
        \item Establece los circulos correspondientes a los bordes del objetos.
        \item Binarización de las imágenes.
        \item Proyección de la seccción del iris.
    \end{itemize}
    \item \textit{extraction} (parte del código adaptado, no utilizado en este proyecto).
    \item \textit{clasification} (parte del código adaptado, no utilizado en este proyecto).
    \item \textit{clasificacionCNN}
    \begin{itemize}
        \item Adaptación del \textit{dataset} al modelo de entrada de la red neuronal.
        \item División del \textit{dataset} para el entrenamiento y el testeo.
        \item Construcción de los tres modelos.
        \item Muestra de resultados y de un ejemplo de clasificación.
    \end{itemize}
\end{enumerate}


\subsubsection{Ejecucción del \textit{pipeline}}

Cada una de las secciones anteriores ha sido encapsulada en una única función, con la configuración del \textit{pipeline} como único parámetro de entrada.
Para ejecutar el \textit{pipeline} se establece cada función dentro de un \textit{FunctionTransformer} y se coloca en el orden requerido dentro de la función \textit{Pipeline}.
Finalmente, se ejecuta la \textit{pipeline}, pasandole la configuración como parámetro.


\begin{lstlisting}[language=Python] 

_1_tratar_dataset_pip = FunctionTransformer(tratar_dataset_casia)
_1_1_data_augmentation_pip = FunctionTransformer(data_augmentation)
_2_1_segmentation_pip = FunctionTransformer(segmentation)
_2_2_normalization_pip = FunctionTransformer(normalization)
_4_clasificationCNN_pip = FunctionTransformer(clasificationCNN)

iris_recognition_pipeline = Pipeline([('_1_tratarDataset', _1_tratar_dataset_pip), 
('_1_1_dataAugmentation', _1_1_data_augmentation_pip), 
('_2_1_segmentation', _2_1_segmentation_pip), 
('_2_2_normalization', _2_2_normalization_pip),
('_3_1_extraction', _3_1_extraction_pip),
('_3_2_clasification', _3_2_clasification_pip), 
('_4_clasificationCNN', _4_clasificationCNN_pip) ])

iris_recognition_pipeline.transform(confi_dict)

\end{lstlisting} 

\subsection{\textit{Notebook} \textit{Accuracy.ipynb}} \label{anx:accuracy}

\section{Compilación, instalación y ejecución del proyecto}

Pasos para la ejecución del proyecto:

\begin{itemize}
    \item Clonar el repositorio del proyecto \url{https://github.com/Ponsoda/tfm-iris-recognition}.
    \item Descargar el \textit{dataset} de CASIA-V1. En el proyecto se ha utilizado el \textit{dataset} a partir de \url{https://github.com/jaa0124/iris_classifier/tree/master/notebooks/CASIA-IrisV1}.
    \item Abrir el \textit{notebook} \texttt{9-pipelines\_v7.ipynb}.
    \item Determinar la ubicación del \textit{dataset} de CASIA-V1 en la configuración del \textit{pipeline}, así como el resto de parámetros y el directorio de salida para los modelos.
    \item Correr todas las celdas del proyecto en el orden definido en el proyecto para la creación de cada uno de los modelos.
    \item Abrir el \textit{notebook} \texttt{Accuracy.ipynb}.
    \item Establecer la ubicación del \textit{dataset} reservado para el testeo y del modelo a utilizar y lanzar todas las celdas del \textit{notebook}.
\end{itemize}

\section{Pruebas del sistema}

Las diferentes pruebas que se han ido realizando están accesible en el directorio en forma de \textit{notebooks} de Jupyter que pueden ser descargados y ejecutados.

%\include{./tex/E_Manual_usuario}


\bibliographystyle{plain}
\bibliography{bibliografia}

\end{document}
