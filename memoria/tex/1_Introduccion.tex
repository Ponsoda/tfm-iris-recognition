\capitulo{1}{Introducción}

El \textit{Oxford Learners Dictionaries} define la biometría como la utilización de características humanas parar poder identificar a las personas.

Esta cualidad de la biometría se encuentra principalmente para los humanos en el reconocimiento facial y del iris y huellas dactilares, y hoy en día
se ha convertido en un elemento fundamental en la seguridad de nuestros dispositivos electrónicos, lo cual supone indirectamente,
que de la biometría depende el acceso a nuestra información privada.

Dentro de la biomtería, el iris se utiliza como elemento de reconocimiento biométrico de gran eficacia, tanto por su inmutabilidad a lo largo del tiempo como por resultar un valor único y 
personal, que supone que dos personas no tendrían un iris idéntico \cite{malgheet_iris_2021}. 

En este estudio, se han comparado técnicas que permitan la identificación de indivíduos a través de imágenes de su iris. 
En concreto, la comparación se ha llevado a cabo, utilizando la precisión de los modelos a la hora de clasificar. Por un lado, la adaptación de una red neuronal 
a el dataset, sin ningún preprocesamiento, y por el otro, esta misma adaptación con un proceso previo de normalización del iris .

Para la primera fase, se ha empleado una red neuronal preentrenada de ImageNet\footnote{ImageNet es un proyecto donde se proporciona una gran base de datos de imágenes para usos no comerciales \url{https://www.image-net.org/}} a la que se le ha aplicado \textit{fine-tuning} con una parte del dataset de imágenes del iris.
Para la segunda fase, a la red neuronal preentrenada, se le ha aplicado \textit{fine-tuning} de un dataset donde el iris ya ha sido normalizado \cite{tfg_iris_2020}.

El objetivo de este proyecto ha sido el de analizar cual de estas perspectivas es más óptima para el reconocimiento de un indivíduo a través de su iris.

\section{Outline}

El resto del documento se estructura de la siguiente manera. El capítulo 2 \nameref{capitulo2} contiene los objetivos del proyecto. El capítulo 3 \nameref{capitulo3} contiene los conceptos teóricos necesarios
para entender el proyecto. El capítulo 4 \nameref{capitulo5} muestra las técnicas y herramientas utilizadas en el desarrollo de este trabajo. En el capítulo 5 \nameref{capitulo5} se muestran los Aspectos
más relevantes que se han desarrollado. En el capítulo 6 \nameref{capitulo6} los trabajos relacionados y en el capítulo 7 \nameref{capitulo7} las conclusiones y las líneas de trabajo futuras.
