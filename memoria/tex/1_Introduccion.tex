\capitulo{1}{Introducción} \label{capitulo1}

El \textit{Oxford Learners Dictionaries} \footnote{Definición consultada en \url{https://www.oxfordlearnersdictionaries.com/definition/english/biometric}} define la biometría como la utilización de características humanas parar poder identificar a las personas. Esta idea se ha convertido en la base o principio fundamental de un número considerable de sistemas de seguridad.

Las principales características utilizadas para identificación de individuos son la cara, el iris, y las huellas dactilares. Estas características identificativas se han convertido en elementos de seguridad fundamentales para multitud de dispositivos. 

Esta dependencia de la biometría para acceder a los dispositivos supone indirectamente una dependencia en ella para la protección de la información privada de la población, ya que hoy en día, los dispositivos electrónicos cuentan con gran cantidad de información sensible.

Dentro de la biometría, el iris se utiliza como un elemento de reconocimiento biométrico de gran eficacia, tanto por su inmutabilidad a lo largo del tiempo como por resultar un valor único y 
personal, que supone que dos personas no puedan ser identificadas con un mismo iris \cite{malgheet_iris_2021}.

Para la utilización del iris como identificador a partir de una imagen ocular el procedimiento habitual es el siguiente. En primer lugar, se lleva a cabo una segmentación y una normalización del iris, para posteriormente extraer las características del iris, seleccionar las más significativas y, finalmente, llevar a cabo la clasificación \cite{malgheet_iris_2021}.

Sin embargo, contar con técnicas eficientes de segmentación y extracción de atributos del iris hoy en día sigue siendo un problema. En muchos casos se siguen utilizando técnicas clásicas que buscan encontrar bordes en el iris para poder aislarlo, como se muestra en la Figura \ref{img:segmentacion-ojo}, y ocurre algo similar con la extracción de características.
No obstante, se ha demostrado que las redes neuronales pueden ser utilizadas para la segmentación del iris \cite{lozej_end--end_2018} y la extracción de características \cite{tfg_iris_2020}\cite{boyd_deep_2020}.

Existen dos opciones a la hora de trabajar con las redes neuronales para la extracción de características: (a) entrenar una red neuronal desde cero, (b) adaptar una ya existente al conjunto de datos a utilizar. Ha sido demostrado que el uso de redes neuronales adaptadas funciona mejor que las creadas desde cero cuando se aísla previamente el iris \cite{boyd_deep_2020}, pero surgen dos nuevas preguntas:

\begin{enumerate}
    \item ¿Es necesario aislar el iris para llevar a cabo la adaptación de la red neuronal, o esta podría llevarse a cabo también de forma eficiente con las imágenes oculares completas?
    \item ¿Una vez que se ha adaptado la red neuronal a el conjunto de datos, sería posible que esta llevase a cabo, no solo la extracción de características sino también la clasificación?
\end{enumerate}

\imagen{img/10_segmentacion_ojo.png}{Segmentación de los bordes límbico y pupilar de forma clásica.}{Segmentación de los bordes límbico y pupilar de forma clásica.}{img:segmentacion-ojo}

En este estudio, se ha adaptado una red neuronal pre-entrenada para la identificación de objetos para que sea capaz de identificar de individuos a través de sus imágenes oculares. Se proponen dos enfoques. Un primer enfoque donde la red neuronal se adapta con el conjunto de imágenes oculares completas y un segundo enfoque donde la red neuronal se adapta con el conjunto de imágenes, previo aislamiento del iris. Ejemplos imágenes pertenecientes a cada conjunto de datos se pueden consultar en la Figura \ref{img:completa-vs-aislada}.   

\imagen{img/18_completo_vs_aislado.png}{Ejemplo de imágenes completas e imágenes con el iris aislado.}{Ejemplo de imágenes completas e imágenes con el iris aislado.}{img:completa-vs-aislada}

En ambos enfoques se ha empleado la red neuronal VGG16\footnote{Esta red neuronal cuenta con 16 capas y ha sido entrenada con más de un millón de imágenes.}, que ha sido pre-entrenada con ImageNet\footnote{ImageNet es un proyecto donde se proporciona 
una gran base de datos de imágenes para usos no comerciales \url{https://www.image-net.org/}}. Esta red ha sido adaptada para clasificar individuos aplicando técnicas de \textit{fine-tuning}\footnote{El \textit{fine-tuning} 
permite adaptar el modelo para que, al llevarse a cabo la clasificación, no muestre resultados relativos al conjunto de datos con el que ha sido entrenada, sino con el que ha sido adaptada, tal como se muestra en la Figura \ref{img:enfoque-sin-normalizacion}. El \textit{fine-tuning} está definido en la sección \ref{subsubsec:segmentacion}.}
 con imágenes oculares etiquetadas. Estos dos enfoques han dado lugar a dos modelos de clasificación. 

En un primer enfoque, se han utilizado las imágenes oculares completas del conjunto de datos\footnote{Revisar sección \ref{casia}} para adaptar la red neuronal, siguiendo el proceso que se muestra en la Figura \ref{img:enfoque-sin-normalizacion}

\imagen{img/14_enfoque_sin_normalizacion.png}{Enfoque utilizando las imágenes sin preprocesamiento.}{Enfoque utilizando las imágenes sin preprocesamiento.}{img:enfoque-sin-normalizacion}


En un segundo enfoque, se ha aplicado un proceso de segmentación\footnote{Definición de segmentación en la sección \ref{subsubsec:segmentacion}.} y normalización\footnote{Definición de normalización en la sección \ref{subsubsec:normalizacion}.} del iris, desarrollado en \cite{tfg_iris_2020}, al conjunto de imágenes, para aislar el iris en la imagen. Posteriormente, se ha seguido también un proceso de adaptación de la red neuronal. Este enfoque se muestra en la Figura \ref{img:enfoque-normalizacion}. 

\imagen{img/13_enfoque_normalizacion.png}{Enfoque utilizando la normalización y segmentación del iris.}{Enfoque utilizando la normalización y segmentación del iris.}{img:enfoque-normalizacion}



Adicionalmente, y teniendo en cuenta que el conjunto de datos es relativamente pequeño (un total de siete imágenes oculares por individuo), se ha querido analizar el efecto de la utilización de un conjunto de datos de mayor tamaño para la adaptación de la red neuronal. Por ello, se han creado dos conjuntos de datos adicionales utilizando técnicas que aumenten artificialmente el número de imágenes por individuo.

Por ello, para ambos enfoques se han creado dos modelos adicionales aplicando técnicas de \textit{data augmentation}\footnote{El término \textit{data augmentation} hace referencia a un conjunto de técnicas que permite ampliar el conjunto de datos original con variaciones de el mismo. En la sección \nameref{dataaugmentation} puede encontrar una explicación más detallada.} para, por un lado, incrementar el número de imágenes oculares por individuo y, por otro, entrenar al modelo con nuevas variaciones de las imágenes, haciendo el modelo más robusto ante estas posibles nuevas variaciones.

El resultado de la adaptación de los cuatro conjuntos de imágenes a red neuronal ha sido la creación de cuatro modelos de clasificación, capaces de identificar a individuos a partir de una imagen ocular. 

Para determinar qué enfoque ha sido capaz de identificar de forma más eficiente a los individuos, se ha comparado la tasa de acierto de los modelos resultantes de las adaptaciones a la hora de clasificar nuevas imágenes oculares. 

El objetivo principal de este proyecto ha sido el de analizar cuál de estos enfoques es el óptimo para el reconocimiento de individuos a través de sus imágenes oculares, así como comprobar el efecto del \textit{data augmentation } y el aislamiento del iris en las tasas de acierto.

\section{Outline}

El resto del documento se estructura de la siguiente manera. El capítulo 2 \nameref{capitulo2} define las principales motivaciones del proyecto. El capítulo 3 \nameref{capitulo3} se concentra en los aspectos teóricos del proyecto. En el capítulo 5 \nameref{capitulo5} se muestran los aspectos
más relevantes que se han desarrollado. En el capítulo 6 \nameref{capitulo6} los trabajos relacionados y en el capítulo 7 \nameref{capitulo7} las conclusiones y las líneas de trabajo futuras.
