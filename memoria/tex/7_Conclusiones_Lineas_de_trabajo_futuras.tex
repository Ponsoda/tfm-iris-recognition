\capitulo{7}{Conclusiones y Líneas de trabajo futuras} \label{capitulo7}

En este apartado se explican las conclusiones así como se establecen las posibles líneas futuras.

\section{Conclusiones}
Como conclusion de este proyecto, la utilización de técnicas de machine learning, contando con ordenadores de pocos recursos, demuestra que el preprocesamiento
es necesario para centrar los procesos en los elementos de las imágenes verdaderamente importantes. Por otro lado, una vez que la imagen está pre-procesada, la utilización 
de redes neuronales con \textit{fine-tuning} para clasificar las imágenes se ha resuelto como un mayor accuracy que el modelo que aplica machine learning en la última fase
pero este último ha demostrado ser más rápido, por lo tanto, la utilización de una u otra técnica variará según los recursos que se tengan y el contexto en el que se 
vaya a utilizar (inmediatez con que se necesiten los resultados).


\section{Líneas de trabajo futuras}

Las líneas de trabajo futuras se podrían determinan con la utilización de estas técnicas con nuevos datasets, consiguiendo un modelo lo suficientemente robusto
que permitiese su utilización en un programa de escritorio, con una primera fase de \textit{fine-tuning} con imágenes del ojo del usuario y una segunda fase donde este 
se utilizase como método de seguridad para el acceso a ciertos documentos de los aparatos electrónicos.