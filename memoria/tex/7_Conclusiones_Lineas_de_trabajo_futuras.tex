\capitulo{7}{Conclusiones y Líneas de trabajo futuras} \label{capitulo7}

En este apartado se explican las conclusiones del proyecto, así como se establecen las posibles líneas de trabajo futuras.

\section{Conclusiones}

Como conclusión de este proyecto, se puede determinar que la utilización de técnicas de \textit{fine-tuning} para llevar a cabo todo el proceso de clasificación,
resultan, en este caso de estudio, más eficientes cuanto mayor es el tamaño de la imagen, y no cuando se aisla la zona del iris, como sse podía pensar al entender
que el iris es, con diferencia, la parte más característica del ojo. 

Por otro lado, la utilización de técnicas de \textit{data augmentation} con el fin de aumentar el tamaño de los \textit{datasets} y permitir la creación de unos mmodelos más robustos,
no a influido positivamente en las tasas de acierto de los modelos, sino todo lo contrario.


\section{Líneas de trabajo futuras}

Las líneas de trabajo futuras se podrían determinan con la utilización de estas técnicas con distintos \textit{datasets}, utilizando imágenes que no estén específicamente preparadas
para este tipo de estudios. Además, sería interesante probar distintos modelos para el \textit{fine-tuning} de los distintos \textit{datasets} así como la utilización de redes neuronales
para llevar a cabo el proceso de segmentación del iris.

El objetivo de todo ellos sería el conseguir un modelo lo suficientemente robusto
que permitiese su utilización en un programa de escritorio, con una primera fase de \textit{fine-tuning} con imágenes del ojo del usuario y una segunda fase donde este 
se utilizase como método de seguridad para el acceso a ciertos documentos de los aparatos electrónicos.