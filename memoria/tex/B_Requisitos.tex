\apendice{Especificación de Requisitos}

\section{Introducción}

En este apartado se especifican los objetivos generales del proyecto, así como cuales son los requisitos para su consecución.

\section{Objetivos generales}

Los objetivos generales son, el ser capaz de asimilar el trabajo de \cite{tfg_iris_2020}, modificar el código para hacerlo reutilizable y, finalmente, ser capaz de 
utilizar técnicas de \textit{fine-tuning} con el conjunto de imágenes sobre una red neuronal ya pre-entrenada.

\section{Catalogo de requisitos}

Los requisitos del proyecto son los siguientes.

\begin{enumerate}
    \item Optimizar del código previo.
    \item Crear los modelos para la clasificación de imágenes oculares.
        \begin{enumerate}
            \item Dividir el conjunto de datos en entrenamiento y test.
            \item Aplicar de \textit{data augmentation}.
            \item Segmentar de las imágenes.
            \item Normalizar de las imágenes.
        \end{enumerate}
    \item Aplicar \textit{fine-tuning} a los distintos \textit{datasets}.
    \item Crear una configuración donde el usuario pueda modificar los parámetros y visualizar los distintos \textit{outputs}.
    \item Determinar qué proceso de \textit{fine-tuning} proporciona una tasa de acierto más alta.
  \end{enumerate}

\section{Especificación de requisitos}

\subsection{Optimizar el código previo}

Asimilar y optimizar el código previo para que sea funcional. Para llevar a cabo este apartado es necesario contar con el código y el conjunto de datos, ambos accesibles desde \url{https://github.com/jaa0124/iris_classifier/}.

\subsection{Crear los modelos para la clasificación de imágenes oculares}

La clasificación de las imágenes oculares se lleva a cabo utilizando modelos adaptados utilizando distintos \textit{datasets}. Su creación se lleva a cabo en las siguientes fases:

\subsubsection{Dividir el conjunto de datos en entrenamiento y test}

División del conjunto de datos de CASIA en entrenamiento y test, para poder validar los modelos. Los requisitos previos son el 1.

\subsubsection{Aplicar de \textit{data augmentation}}

Aplicación de \textit{data augmentation} sobre dos de los modelos. Los requisitos previos son el 1 y el 2a. Este paso solo se aplica a dos de los \textit{datasets}.

\subsubsection{Segmentar de las imágenes}

Segmentación de las imágenes donde el iris será aislado. Los requisitos previos son 1, 2a y 2b. Este paso solo se aplica a dos de los \textit{datasets}.

\subsubsection{Normalizar de las imágenes}

Normalización de las imágenes donde el iris será aislado. Los requisitos previos son 1, 2a, 2b y 2c. Este paso solo se aplica a dos de los \textit{datasets}.

\subsection{Aplicar \textit{fine-tuning} a los distintos \textit{datasets}}

Aplicación de \textit{fine-tuning} sobre los datasets. Los requisitos previos son  Los requisitos previos son 1, 2a, 2b, 2c y 2d. Dependiendo del conjunto de datos, alguno de los pasos previos puede no ser necesario.

\subsection{Crear una configuración donde el usuario pueda modificar los parámetros y visualizar los distintos \textit{outputs}}

Creación de una \textit{pipeline} y establecimiento de su configuración para permitir al usuario adaptar el código a sus necesidades. Todos los pasos anteriores son requisitos previos.

\subsection{Determinar qué proceso de \textit{fine-tuning} proporciona una tasa de acierto más alta}

Determinación de los valores más altos de la tasa de acierto en los distintos modelos.



