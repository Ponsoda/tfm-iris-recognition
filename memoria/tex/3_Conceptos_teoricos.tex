\capitulo{3}{Conceptos teóricos} \label{capitulo3}

En esta sección se explican los principales conceptos teóricos relacionados con el proyecto, relacionados con la biometría, la inteligencia
 artificial y el \textit{fine-tuning}.


\section{Biometría}

Como se ha comentado en \nameref{capitulo1}, la biometría permite la identificación de un individuo a través de determinadas características que 
se asocian a su persona. 

De entre todos los elementos biométricos, que incluyen huellas, cara, iris o voz, el \textit{iris recognition system} (IRS) es el método con mayor eficiencia a la hora de determinar la identidad
de las personas \cite{malgheet_iris_2021}, ya que el iris es el mismo a lo largo de la vida de una persona y es único, incluso entre gemelos. Esto es incluso utilizado en los procesos 
post-mortem para poder determinar la pertenencia del cuerpo \cite{boyd_post-mortem_2020}. El iris humano es un órgano del ojo, que tiene como función, controlar el tamaño
de la pupila en función de la cantidad de luz que llega hasta este \cite{boyd_post-mortem_2020}.

\subsection{\textit{Dataset} CASIA-IrisV1 }	\label{casia}

Se trata de una base de datos que contiene 756 imágenes del iris de un total de 108 sujetos. 
Dichas fotos fueron tomadas por el \href{http://www.cbsr.ia.ac.cn/english/index.asp}{Center for Biometrics and Security Research} en dos sesiones, donde se tomaron 3 y 4 muestras respectivamente por cada individuo, con una resolución de 320x280. 
La pupila fue automáticamente remplazada para evitar que en ella se reflejasen las luces de las fotografías, tal como podemos observar en la figura \ref{img:ojo-reflejo} 
\footnote{El proceso de la toma de muestras se describe en \url{http://www.cbsr.ia.ac.cn/IrisDatabase.htm}}.

\imagen{img/02_ojo_raw_vs_ojo_reflejo.png}{Eliminación del reflejo de la pupila \cite{tfg_iris_2020}.}{Eliminación del reflejo de la pupila.}{img:ojo-reflejo}
\section{Inteligencia Artificial}

La Real Academia Española define la inteligencia artificial como una disciplina cuyo objetivo principal es la creación de programas capaces de realizar funciones similares a los de la mente humana \footnote{En base a la definición de \url{https://dle.rae.es/inteligencia}.}.

\subsection{\textit{Deep Learning}}

El \textit{deep learning} es un tipo de \textit{machine learning}, que utilizando redes neuronales, intenta replicar la forma en la que los humanos toman las decisiones. \footnote{En base a la definición de \url{https://www.ibm.com/cloud/learn/deep-learning}.}

\subsection{\textit{Pipeline}}

Los \textit{pipelines} son secuencias de procesos encadenados, donde el \textit{output} del proceso anterior funciona como \textit{input} del siguiente proceso. La utilización de \textit{pipelines} ha permitido un mayor control de los distintos procesos, al poder 
establecer una configuración general para la cadena de procesos, y para poder modificar de una maner más sencilla el orden de las procesos, a la hora de experimentar con nuevas combinaciones.

\subsection{\textit{Data augmentation}}\label{dataaugmentation}

El \textit{data augmentation} es un proceso común en el análisis de imágenes, y en aquellos proyectos donde se utilicen procesos estadísticos. 

Este consiste en aumentar el tamaño de los \textit{dataset} con la creación artificial de nuevas imágenes, que son producidas a partir de imágenes del \textit{dataset} original.
Para ello, la imagen original se modifica, comúnmente con la aplicación de ruido gaussiano o transformaciones geométricas de tipo afín, teniendo como resultado, una imagen que deriva de la original, pero que cuenta con ciertas diferencias, que serán más o menos pronunciada dependiendo de las técnicas 
de \textit{data augmentation} que se le apliquen, así como de los valores utilizados para realizar las modificaciones \footnote{Por ejemplo, si se rota una imagen 2 grados, la diferencia con la original será mucho menor que si se rota 180 grados.}.	

En el caso de los procesos de entrenamiento de las redes neuronales es común
la utilización de técnicas de data augmentation, principalmente por dos situaciones, aunque estas no son limitantes:

\begin{itemize}
	\item Número insuficiente de datos: en este caso, el data augmentation se aplica porque el dataset no es lo suficientemente grande como para conseguir unos resultados 
positivos en la creación de una red neuronal.
	\item Aumento de la robustez del modelo: el segundo supuesto principal por el cual se utiliza data augmentation es la utilización de elementos que añadan dificultades a la
red neuronal para cumplir su propósito, lo cual permitirá una mayor robustez del modelo.
\end{itemize}

\subsubsection{Ruido gaussiano}

La primera de las técnicas de data augmentation utilizadas ha sido el ruido gaussiano, también conocido como ruido blanco. 
Este técnica provoca que los píxeles de una imagen cambien su valor siguiendo una distribución gaussiana, como se puede observar en la figura \ref{img:ruido-gaussiano} .

\imagen{img/03_rudio_gaussiano.png}{Ejemplo de \textit{data augmentation} por ruido gaussiano.}{Ejemplo de \textit{data augmentation} por ruido gaussiano.}{img:ruido-gaussiano}

\subsubsection{Transformaciones afines}

Las transformaciones afines permiten aumentar el tamaño del dataset mediante la transformación de imágenes, donde conservan el paralelismo de sus líneas rectas y paralelas y de alguna forma, simulan una nueva perspectiva de la imagen original.
En cuanto a los tipos de transformaciones afines encontramos la transformación de identidad, reflexión, escalamiento, traslación y finalmente, la rotación, como puede comprobarse en la tabla \ref{tabla:transformaciones-afines}.

\begin{table}[h!]
\begin{tabular}{ |p{2cm}|p{1.8cm}|p{2.2cm} |p{2cm}|p{1.8cm}|}
	\hline
	\multicolumn{5}{|c|}{Transformaciones afines}\\ 
	\hline
	Identidad & Reflexión &Escalamiento & Traslación  & Rotación\\
	\hline
	Figura \ref{img:transformacion-identidad} & Figura \ref{img:transformacion-reflejo}     & Figura \ref{img:transformacion-escalamiento} &   Figura \ref{img:transformacion-traslacion} &   Figura \ref{img:transformacion-rotacion}\\
	\hline
\end{tabular}
\caption{\label{tabla:transformaciones-afines}Transformaciones afines aplicadas en el trabajo.}
\end{table}


\paragraph{Identidad}

La transformación de identidad es un tipo de transformación afín en el que la imagen se copia sin ningún otro cambio, y se utiliza para la  
la reutilización de los datasets.

\imagen{img/05_transformacion_identidad.png}{Ejemplo de transformacion de identidad.}{Ejemplo de transformacion de identidad.}{img:transformacion-identidad}

\begin{gather}
	Identidad:
	\begin{bmatrix} x' \\ y' \\ 1 \end{bmatrix}
	\equiv
	 \begin{bmatrix}
	  1 & 0 & 0 \\
	  0 & 1 & 0 \\
	  0 & 0 & 1 \\
	  \end{bmatrix}
	  \begin{bmatrix} x \\\ y \\ 1 \end{bmatrix}
\end{gather}

%https://subscription.packtpub.com/book/data/9781789537147/1/ch01lvl1sec04/applying-affine-transformation

\paragraph{Reflexión}

Se trata de un mapeo aplicado a la misma imagen a partir de un eje.

\imagen{img/06_transformacion_reflejo.png}{Ejemplo de transformacion por reflexión.}{Ejemplo de transformacion por reflexión.}{img:transformacion-reflejo}

\begin{gather}
	Reflexión:
	\begin{bmatrix} x' \\ y' \\ 1 \end{bmatrix}
	\equiv
	 \begin{bmatrix}
	  1 & 0 & 0 \\
	  0 & -1 & 0 \\
	  0 & 0 & 1 \\
	  \end{bmatrix}
	  \begin{bmatrix} x \\\ y \\ 1 \end{bmatrix}
\end{gather}

\paragraph{Escalamiento} 

Esta transformación modifica la escala de la imagen original, ya sea ampliándola o disminuyéndola.

\imagen{img/07_transformacion_escala.png}{Ejemplo de transformacion por escalamiento.}{Ejemplo de transformacion por escalamiento.}{img:transformacion-escalamiento}


\begin{gather}
	Escalamiento:
	\begin{bmatrix} x' \\ y' \\ 1 \end{bmatrix}
	\equiv
	 \begin{bmatrix}
	  Sx & 0 & 0 \\
	  0 & Sy & 0 \\
	  0 & 0 & 1 \\
	  \end{bmatrix}
	  \begin{bmatrix} x \\\ y \\ 1 \end{bmatrix}
\end{gather}

\paragraph{Traslación}

La imagen cambia de plano de coordenadas, pero no se modifican ni su tamaño, ni su forma ni su orientación.

\imagen{img/08_transformacion_traslacion.png}{Ejemplo de transformacion por traslación.}{Ejemplo de transformacion por traslación.}{img:transformacion-traslacion}

\begin{gather}
	Traducción:
	\begin{bmatrix} x' \\ y' \\ 1 \end{bmatrix}
	\equiv
	 \begin{bmatrix}
	  1 & 0 & dx \\
	  0 & 1 & dy \\
	  0 & 0 & 1 \\
	  \end{bmatrix}
	  \begin{bmatrix} x \\\ y \\ 1 \end{bmatrix}
\end{gather}

\paragraph{Rotación}

Esta transformación aplica una transformación de $\theta$ grados del plano.

\imagen{img/09_transformacion_rotacion.png}{Ejemplo de transformacion por rotación.}{Ejemplo de transformacion por rotación.}{img:transformacion-rotacion}


\begin{gather}
	Rotación:
	\begin{bmatrix} x' \\ y' \\ 1 \end{bmatrix}
	\equiv
	 \begin{bmatrix}
	  \cos(\theta) &-\sin(\theta) & 0 \\
	  \sin(\theta) & \cos(\theta) & 0 \\
	  0 & 0 & 1 \\
	  \end{bmatrix}
	  \begin{bmatrix} x \\\ y \\ 1 \end{bmatrix}
\end{gather}


\subsection{Preprocesamiento}

El preprocesamiento es la manipulación de los datos para que estos tengan el formato requerido para llevar a cabo su procesamiento.

En el caso de este proyecto, la fase de preprocesamiento es la fase en la que se extrae el iris de la imagen, puesto que, tal como indican diferentes estudios \cite{tfg_iris_2020} \cite{abdullah_iris_2015} \cite{malgheet_iris_2021} \cite{boyd_post-mortem_2020} \cite{liu_efficient_2021} \cite{szymkowski_iris-based_2021} \cite{lozej_end--end_2018}, la parte del iris es la que 
permite identificar a las personas.

\subsubsection{Segmentación} \label{subsubsec:segmentacion}

Detección de los bordes \textit{límbico} y \textit{pupilar} utilizando el detector de bordes de Canny \cite{4767851}. Estos bordes son clave para el aislamiento del iris \cite{tfg_iris_2020}, tal como se observa en la figura \ref{img:segmentacion-ojo}.

\imagen{img/10_segmentacion_ojo.png}{Ejemplo de segmentación del ojo extraido de \cite{tfg_iris_2020}.}{Ejemplo de segmentación del ojo.}{img:segmentacion-ojo}


Una vez detectados los bordes, se procede a una binarización de la imagen \ref{img:segmentacion-binarizacion-ojo}, de forma que quede clara la división entre iris y resto del ojo.

\imagen{img/11_segmentacion_binario_ojo.png}{Ejemplo de binarización del ojo durante la segmentación.}{Ejemplo de binarización del ojo durante la segmentación.}{img:segmentacion-binarizacion-ojo}


\subsubsection{Normalización}\label{subsubsec:normalizacion}

Se puede definir como la proyección del iris a coordenadas polares (figura \ref{img:normalizacion-ojo}), utilizando el método Daugman \cite{daugman_normalization_1993}, de manera que se igualen los tamaños de las diferentes imágenes y permitan su comparación.

\imagen{img/12_normalizacion_ojo.png}{Ejemplo de normalización del ojo.}{Ejemplo de normalización del ojo.}{img:normalizacion-ojo}


\subsection{\textit{Fine-tuning}}\label{subsec:fine-tuning}
 Se trata de un método que permite adaptar redes neuronales previamente entrenadas con datos diferentes para un conjunto de datos personalizados. En el caso de este proyecto,
 el \textit{fine-tuning} ha sido utilizado para obtener un clasificador de imágenes para el dataset de CASIA sin tener que crear una red neuronal desde cero, sino adaptando la red
  neuronal VGG16.

