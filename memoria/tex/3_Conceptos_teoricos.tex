\capitulo{3}{Conceptos teóricos} \label{capitulo3}

Explicación de la sección


\section{Biometría}

De entre todos los elementos biometricos, que incluyenn huellas, cara, iris o voz, el \textit{iris recognition system} (IRS) es el metodo con mayor eficiencia a la hora de determinar la identidad
de las personas \cite{malgheet_iris_2021}, ya que el iris es el mismo a lo largo de la vida de una persona y es único, incluso entre gemelos. Esto es incluso utilizado en los procesos 
post-mortem para poder determinar la pertenencia del cuerpo \cite{boyd_post-mortem_2020}. El iris humanos es un organo del ojo, que tiene como funciona controlar el tamaño
de la pupila en función de la cantidad de luz que llega hasta este \cite{boyd_post-mortem_2020}.

\subsection{Análisis del Iris}	

\subsubsection{Dataset CASIA}	

El dataset de CASIA hace referencia a una base de datos que contine 756 imágenes del iris de un total de 108 sujetos. 
Dichas fotos se han realizado en dos sesiones, donde se tomaron 3 y 4 muestras respectivamente por cada indivíduo. 
La pupila fue automáticamente remplazada para evitar que se reflejasen las luces de las fotografías 
\footnote{El proceso de toma de muestras se describe en \url{http://www.cbsr.ia.ac.cn/IrisDatabase.htm}}.

\imagen{img/02_ojo_raw}{Ejemplo de una imagen del dataset CASIA}

\section{Inteligencia Artificial}

La Real Academia Española define la inteligencia artificial como una "Disciplina científica que se ocupa de crear programas informáticos que ejecutan 
operaciones comparables a las que realiza la mente humana, como el aprendizaje o el razonamiento lógico."

\subsection{Deep Learning}

El deep learning intenta replicar la forma en la que los humanos toman las decisiones en las máquinas a través de redes neuronales.

\subsubsection{Data augmentation}

El data augmentation es un proceso común en el análisis de imágenes y de datos en general. 

Consiste en aumentar el tamaño de dataset con nuevas imágenes producidas a partir del dataset original, modificando la misma para conseguir un modelo más robusto.	

En el caso de los procesos de entrenamiento de las redes neuronales es común
la utilización de técnicas de data augmentation, principalmente por dos situaciones, aunque estas no son excluyentes:

\begin{itemize}
	\item Número insuficiente de datos: en este caso, el data augmentation se aplica porque el dataset no es lo suficientemente grande como para conseguir unos resultados 
positivos en la creación de una red neuronal.
	\item Aumento de la robustez del modelo: el segundo supuesto principal por el cual se utiliza data augmentation es la utilización de elementos que añadan dificultades a La
red neuronal para cumplir su propósito, lo cual permitirá una mayor robustez del modelo.
\end{itemize}

\subsubsection{Preprocesamiento}

El preprocesamiento es la manipulación de los datos para que estos tengan el formato requerido para llevar a cabo su procesamiento.

En el caso de este proyecto, la fase de preprocesamiento es la fase en la que se extrae el iris de la imagen, puesto que, tal como indican diferentes estudios (referencia), la parte del iris es la que 
permite identificar a las personas.

\subsubsection{Segmentación}

Detección de las distintas zonas del ojo.

\subsubsection{Normalización}

Normalización del iris y desenrollamiento en coordenadas polares

\subsection{Ruido gaussiano}

La primera de las técnicas de data augmentation utilizadas ha sido el ruido gaussiano, también conocido como ruido blanco. Este ruido provoca que los píxeles de una imagen cambien
su valor siguiendo una distribución gausseana.

\subsection{Transformaciones afines}

Las transformaciones afines son transformaciones de las imágenes que conservan el paralelismo de sus líneas rectas y paralelas y de alguna forma simulan una nueva perspectiva para esta.
En cuanto a los tipos de transformaciones afines encontramos las siguientes: transformación de identidad, escalamiento, traducción, inclinación (de X o Y), rotación.

\subsubsection{Transformación de identidad}

\subsubsection{Escalamiento}

\subsubsection{Traducción}

\subsubsection{Inclincación}

\subsubsection{Rotación}

\begin{tabular}{ |p{2.5cm}|p{2.5cm}|p{2.5cm}|p{2.5cm}|p{2.5cm}|  }
	\hline
	\multicolumn{5}{|c|}{Transformaciones afines}\\ 
	\hline
	Identidad & Escalamiento & Traducción & Inclinación & Rotación\\
	\hline
	Imagen   & Imagen    &Imagen&   Imagen &   Imagen\\
	\hline
   \end{tabular}

\section{Fine tunning}
   El fine tunning es una técnica que se utiliza para poder ajustar los parámetros de las redes neuronales preentrenadas a los necesarios en relación con
   el dataset con el que se trabaja.