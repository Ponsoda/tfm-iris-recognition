\capitulo{4}{Técnicas y herramientas} \label{capitulo4}

En esta sección se describen los instrumentos y recursos con los cuales se ha llevado a cabo el proyecto. 

\section{Hardware}

Para la ejecución de los notebooks de este proyecto se ha contado con una máquina de la UBU, cuyas características se recogen en la tabla \ref{caracteristicaspc}.

\begin{table}
\begin{tabular}{l c}
Elemento & Característica \\
\hline
Procesador & Intel Xeon E5-2630 v4 @ 2.20GHz (4 núcleos)\\
Memoria & 128GB\\
GPUs & 3 x Titan XP\\
Discos & SSD 500Gb, 2 x HDD 2TB\\
\end{tabular}
\caption{\label{caracteristicaspc} Características del equipo}
\end{table}

Para interactuar con la máquina, se ha utilizado una conexión ssh \footnote{Protocolo que permite el acceso remoto a través de un canal seguro \url{https://www.openssh.com/}.} así como putty, para la modificación de los notebooks a través de Jupyter notebooks, dentro del ecosistema de Anaconda.

\section{Github}
Para el control de versiones y el seguimiento de fases del proyecto

\section{Python}

Además se ha utilizado python para todo el proyecto, esto quiere decir pre-procesado, creación y utilización de redes neuronales y clasificación con técnicas de 
machine learning.

Entre las principales librerías utilizadas se encuentran:

\begin{itemize}
    \item os, para el acceso a los directorios
    \item numpy, para trabajar con las imágenes a nivel de arrays
    \item scikit-image, para la transformación de las imágenes y el uso de dataset
    \item tensorflow, para la modificación de las redes neuronales
    \item keras, para el manejo de las redes neuronales
    \item matplot, para las gráficas
\end{itemize}

\subsection{Redes neuronales pre-entrenadas}

Así mismo, se ha utilizado la red neuronal pre-entrenada, basada en U-Net y accesible desde \url{https://github.com/jus390/U-net-Iris-segmentation}, la cual ya había sido
entrenada para la segmentación del iris.

Finalmente, para el último proceso del proyecto, se utiliza imagenet como red neuronal pre-entrenada central.

\subsection{Visual Studio}

Tanto para la redacción de la memoria con latex como para el la creación de los notebooks.

\section{Scrum}
Scrum es un marco que ayuda a la organización de los equipos entorno a un proyecto, en base a Sprints de una determinada duración y permite una retroalimentación
continua del proyecto, de forma que se asegura que el equipo trabaje en consonancia.

En el caso de este proyecto, se ha adecuado la metodología de Scrum para cuadrar reuniones semanales o bisemanales con los tutores del proyecto, estableciendo
los sprints y los objetivos de cada sprint utilizando la plataforma Github.





