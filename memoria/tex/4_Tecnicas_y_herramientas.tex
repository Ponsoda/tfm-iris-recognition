\capitulo{4}{Técnicas y herramientas} \label{capitulo4}

En esta sección se describen los instrumentos y recursos utilizados a lo largo del proyecto. 

\section{Hardware} \label{Hardware}

Para la ejecución de los \textit{notebooks}, se ha contado con una máquina de la Universidad de Burgos, cuyas características se recogen en la tabla \ref{caracteristicaspc}.

\begin{table}[h!]
\begin{tabular}{l c}
Elemento & Característica \\
\hline
Procesador & Intel Xeon E5-2630 v4 @ 2.20GHz (4 núcleos)\\
Memoria & 128GB\\
GPUs & 3 x Titan XP\\
Discos & SSD 500Gb, 2 x HDD 2TB\\
\end{tabular}
\caption{\label{caracteristicaspc} Características del equipo}
\end{table}

Para interactuar con la máquina, se ha utilizado una conexión SSH \footnote{El \textit{security shell} es un protocolo que permite el acceso remoto a través de un canal seguro \url{https://www.openssh.com/}.} así como un cliente Putty \footnote{Putty es una implementación libre de SSH para Windows \url{https://www.putty.org/}.}, para la modificación de los \textit{notebooks} a través de Jupyter \textit{notebooks} \footnote{Se trata de una aplicación que permite editar y lanzar \textit{notebooks} a través del buscador, \url{https://jupyter.org/}.}, dentro del ecosistema de Anaconda \footnote{Anaconda es una distribución libre, accesible a través del buscador, que es ampliamente utilizada en la ciencia de datos \url{https://www.anaconda.com/}.}.

\section{Github}

Github es una compañía que ofrece repositorios Git\footnote{Sistema de control de versiones \url{https://git-scm.com/}.} en la nube. 
Estos repositorios se han utilizado en el proyecto para:

\begin{enumerate}
    \item Control de versiones.
    \item Seguimiento de las fases del proyecto.
    \item Documentación de las reuniones y de los problemas a resolver.
\end{enumerate}

El repositorio del proyecto es accesible desde \url{https://github.com/Ponsoda/tfm-iris-recognition}.

\section{Python}

Se ha utilizado el lenguaje de programación \href{https://www.python.org/}{Python} para las distintas fases del proyecto.

Entre las principales librerías utilizadas se encuentran:

\begin{itemize}
    \item \href{https://imageio.readthedocs.io/en/stable/}{imageio} - leer y escribir imágenes.
    \item \href{https://keras.io/}{keras} - manejo de las redes neuronales.
    \item \href{https://matplotlib.org/}{matplot} - visualización de datos e imágenes.
    \item \href{https://numpy.org/}{numpy} - trabajo con las imágenes a nivel de arrays.
    \item \href{https://docs.python.org/3/library/os.html}{os} - acceso a los directorios.
    \item \href{https://opencv.org/}{opencv} - trabajo con las imágenes.
    \item \href{https://scikit-image.org/}{scikit-image} - transformación de las imágenes y el uso de dataset.
    \item \href{https://docs.python.org/3/library/shutil.html}{shutil} - copia de directorios.
    \item \href{https://www.tensorflow.org/}{tensorflow} - modificación de las redes neuronales.
\end{itemize}

\subsection{Redes neuronales}

En el desarrollo del proyecto, se ha utilizado dos redes neuronales. En primer lugar, durante el proceso de adecuación del código procedente de \cite{tfg_iris_2020}, se ha utilizado una red neuronal basada en U-Net\footnote{Red neuronal para la segmentación de imágenes biomédicas desarrollada por la Universidad de Freiburg \url{https://arxiv.org/abs/1505.04597}.} y accesible desde el \href{https://github.com/jus390/U-net-Iris-segmentation}{repositorio} del \textit{paper} \cite{lozej_end--end_2018}. Esta red neuronal, ya había sido
específicamente entrenada para la segmentar el iris en imágenes oculares. Al tratarse de una red neuronal ya adaptada para dicho fin, se pudo utilizar directamente para la fase de extracción.

Para el proceso de \textit{fine-tuning}, se ha utilizado la red neuronal VGG16, entrenada con el \textit{dataset} de ImageNet para el reconocimiento de objetos, ya descrita en la sección de \nameref{capitulo1}.

\subsection{Visual Studio Code}

Tanto para la redacción de la memoria con \LaTeX como para el la creación de los \textit{notebooks} a nivel local, se ha utilizado \href{https://code.visualstudio.com/}{Visual Studio Code}.

Se trata de un IDE\footnote{Entorno de desarrollo integrado.} con licencia \textit{open-source}, desarrollado por Microsoft que permite funciones de desarrollo, como la edición de código o su depuración.

\section{Scrum}
Scrum es un marco que ayuda a la organización de los equipos entorno a un proyecto, en base a \textit{sprints} de una determinada duración y permite una retroalimentación
continua del proyecto, de forma que se asegura que el equipo trabaje en consonancia.

En el caso de este proyecto, se ha adecuado la metodología de Scrum para cuadrar reuniones semanales o bisemanales con los tutores del proyecto, estableciendo
los sprints y los objetivos de cada sprint utilizando la plataforma Github.





