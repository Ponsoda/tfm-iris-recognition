\apendice{Documentación técnica de programación}

\section{Introducción}

En este apartado se determinará la forma de proceder para poder replicar el proyecto.

\section{Estructura de directorios}

\begin{itemize}
    \item 06\_Models contiene los modelos finales
    \item img contiene las imágenes utilizadas en el propio repositorio
    \item memoria, incluye la memoria
    \begin{itemize}
        \item tex, contiene los archivos tex
        \item img, contiene las imágenes del proyecto
    \end{itemize}
        \item src, incluye el código del proyecto con distintas versiones
\end{itemize}

\section{Manual del programador}


Los \textit{notebooks} del proyecto se pueden explorar en cualquier plataforma compatible con Jupyter notebooks,


\section{Compilación, instalación y ejecución del proyecto}

Pasos para la ejecución del proyecto:

\begin{itemize}
    \item Clonar el repositorio del proyecto \url{https://github.com/Ponsoda/tfm-iris-recognition}.
    \item Descargar el \textit{dataset} de CASIA-V1. En el proyecto se ha utilizado el \textit{dataset} a partir de \url{https://github.com/jaa0124/iris_classifier/tree/master/notebooks/CASIA-IrisV1}.
    \item Abrir el \textit{notebook} \texttt{9-pipelines\_v7.ipynb}.
    \item Determinar la ubicación del \textit{dataset} de CASIA-V1 en la configuración del \textit{pipeline}, así como el resto de parámetros y el directorio de salida para los modelos.
    \item Correr todas las celdas del proyecto en el orden definido en el proyecto para la creación de cada uno de los modelos.
    \item Abrir el \textit{notebook} \texttt{Accuracy.ipynb}.
    \item Establecer la ubicación del \textit{dataset} reservado para el testeo y del modelo a utilizar y lanzar todas las celdas del \textit{notebook}.
\end{itemize}

\section{Pruebas del sistema}

Las diferentes pruebas que se han ido realizando están accesible en el directorio en forma de \textit{notebooks} de Jupyter que pueden ser descargados y ejecutados.
