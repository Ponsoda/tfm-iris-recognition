\apendice{Especificación de diseño}

\section{Introducción}

A continuación se presentan las características asociadas al diseño del proyecto, con el fin de entender mejor como se han construido las distintas partes del proyecto.

\section{Diseño de datos}

\subsection{conjunto de datos} \label{anx:dataset}

Tal como se ha explicado en la memoria, se ha utilizado el conjunto de datos CASIA-V1.

La estructura de datos es la siguiente, por persona y sesión. Figura \ref{img:casia-dir}.

\imagenPequenya{img/17_CASIA_dir.png}{Estructura de directorios de CASIA-V1.}{Estructura de directorios de CASIA-V1.}{img:casia-dir}

\subsection{Jupyter \textit{Notebook}}

El código del proyecto se ha reducido a dos documentos de Jupyter \textit{notebook}:


\begin{itemize}
    \item El primero es el, \texttt{pipeline.ipynb}, en el cual se ha reducido todas las secciones del proyecto, encapsuladas en una \textit{pipeline}.
    \item El segundo es el, \texttt{accuracy.ipynb}, donde se lleva a cabo el cálculo de la tasa de acierto de los modelos.
\end{itemize}


\section{Diseño procedimental}

\begin{enumerate}
    \item Carga de datos
    \item \textit{Data augmentation}
    \item Segmentación y clasificación
    \item \textit{Fine-tuning}
    \item Tasa de acierto
    \end{enumerate}

\section{Diseño arquitectónico}

Con la utilización de un \textit{pipeline} configurable, se ha pretendido que el código utilizado en este proyecto pueda ser re-utilizado y permita modificar su configuración de una manera más sencilla.
