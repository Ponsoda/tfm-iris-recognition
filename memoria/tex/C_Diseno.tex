\apendice{Especificación de diseño}

\section{Introducción}

Se detalla el diseño del proyecto.

\section{Diseño de datos}

\subsection{\textit{Dataset}} \label{anx:dataset}

Tal como se ha explicado en la memoria, se ha utilizado el \textit{dataset} CASIA-V1.

La estructura de datos es la siguiente, por persona y sesión. Imagen \ref{img:casia-dir}.

\imagen{img/17_CASIA_dir.png}{Estructura de directorios de CASIA-V1.}{Estructura de directorios de CASIA-V1.}{img:casia-dir}

\subsection{\textit{Notebooks}}

El código del proyecto se ha reducido a dos \textit{notebooks}:


\begin{itemize}
    \item El primero es el, \texttt{9-pipelines\_v7.ipynb}, en el cual se ha reducido todas las secciones del proyecto, encapsuladas en una \textit{pipeline}.
    \item El segundo es el, \texttt{Accuracy.ipynb}, donde se lleva a cabo el cálculo de la tasa de acierto de los modelos.
\end{itemize}


\section{Diseño procedimental}

\begin{enumerate}
    \item Carga de datos
    \item \textit{Data augmentation}
    \item Segmentación y clasificación
    \item \textit{Fine-tuning}
    \item Tasa de acierto
    \end{enumerate}

\section{Diseño arquitectónico}

Con la utilización de un \textit{pipeline} configurable, se ha intentado que el código sea reutilizable y fácil de manejar.
