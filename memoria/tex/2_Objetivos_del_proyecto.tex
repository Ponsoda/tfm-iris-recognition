\capitulo{2}{Objetivos del proyecto} \label{capitulo2}

En este apartado se detallan los objetivos principales para el desarrollo del proyecto:
\begin{itemize}
    \item Al re-utilizar los procesos de segmentación y normalización de \cite{tfg_iris_2020}, un primer objetivo ha sido optimizar el código ya existente para 
    mejorar su reproducibilidad y que ello permitiera una mayor flexibilidad a la hora de ejecutar el mismo.
    \item El segundo objetivo ha sido el de utilizar técnicas de \textit{fine-tuning} con el fin de adaptar los resultados de clasificación de una red neuronal pre-entrenada al \textit{dataset} utilizado en el proyecto, y comprobar la capacidad de la red adaptada para identificar individuos a partir de sus imágenes oculares.
    \item El tercer objetivo ha sido la aplicación de técnicas de \textit{data augmentation} para comprobar si su utilización supone una mejora en la robustez del modelo.
    \item Como objetivo final, analizar la capacidad de identificación de los las redes neuronales desarrolladas y el impacto que tienen, tanto el aislamiento del iris como el \textit{ata augmentation} en el las identificaciones de individuos a partir de su imagen ocular.                                                                                                                                                                                                           
\end{itemize}

%icons utilizados provenientes de https://www.flaticon.com/
%https://app.diagrams.net/#G12aGVzro-nFg-2UQQPGRj_2C1mn4Qx38x