\capitulo{6}{Trabajos relacionados}

El principal apartado anterior se puede  encontrar en el TFG de extracción del iris, en el cual se basa este trarbajo, puesto se realiza a grandes rasgos
todo lo relativo a la primera de las opciones del trabajo.

Sobre temas de extracción del iris encontramos 02 iris wavelet neural, donde se hace un preprocesamiento con extración del iris utilizando Hough Transform y la 
normalización con Daugmands rubber. Luego, tras eliminar el ruido, la extración se realiza con transformaciones de wavelet. Finalmente, se crea una red neutornal
utilizando el mean-squared error para calcular los pesos en la red.

En 03 deep iris encontramos el desarrollo de técnicas de deep learning para el reconocimiento del iris basado en convolutional neural network residual. utilizando una red preentrenada 
de ResNet50 y fine-tuning, entrenado con una cross-entropy loss function (but they are not using data augmentation, they are using another dataset IIT Delhi dataset and they are 
not doing the preprocessing step).

In 13 ImageNet Deep CNN they use the ImageNet dataset with data augmentation, dropout to train a nerual network to detect images of the feed (maybe we can remove it as
it is not related with iris).

En 14 experimental deep convolutional iris recognition, utilizaron el dataset CASIA - 10000 y la arquitectura VGG-Net, lo cual realiza un PCA para extraer los elementos
más característicos de las imágenes . Después utilizan algoritmos de clasificación para clasificar las imágenes, como el SVM (esto es similar al TFG) y consiguen unos percentajes
de reconocimiento muy altos.

05 techniques review habla de siete pasos en los que se divide un sistema de reconocimiento del iris, 1)adquisición, 2)preprocesamiento, 3)segmentación
 4)normalización, 5)extracción de características, 6) selección de features únicos y característicos, 7)clasificación. Este paper también describe una falta de
 trabajos entorno a datasets de baja cualidad (revisar para sacar más papers). El paper también realza que los sistemas de reconocimiento del iris (IRS) se vuelven 
 poco efectivos cuando las imagenes tienen rotaciones or reflejos, algo que intentamos de mejorar en nuestro proceso, añadiendo ruido con el data augmentation.

 Este mismo paper también comenta los distintos dataset utilizados para estos estudios de reconocimiento de iris, el tipo de ruido utilizado así como su método,
 los tipos de segmentación tradicional y actual utilizados (habitualmente con redes neuronales), técnicas de normalización y extracciñon de caracterñisticas, así 
 como los tipos de accuracy de los métodos de iris recognition.

 De Marsico et al. [44] utiliza también el dataset de casia V! (parece que también Susitha and Subban [81]) para medir el accuracy y Lozej et al. [176] junto con Unet para El
 iris segmentation.

(esto casi se podría quitar pues no es tanto el foco de nuestro trabajo) Varkarakis et al. [179] también utiliza una cnn para segmentar el iris

 Bakshi et al. [72] utiliza filtro gausseano combinado con Hough detección de líneas pero nadie utiliza solo gausseano y/o transformaciones afines (al menos en este
 recopilatorio).

 15 finetunning or raw utiliza técnicas de deep learning para clasificar imagenes de ojos de el casia iris 300 dataset, no utiliza data augmentation. Para segmentation utilizan una herramienta
 llamada OSIRIS y prueban deep learning, fine tunning y raw para ver que clasifica mejor. ( It is better to take the best-performing model trained on either general-purpose or face images and
  fine-tune it to iris recognition task, rather than train own network)
 from scratch.
 

 Existen varios papers que utilizan fully CNN, y lo mismo para feature extraction.



.... (utilizando Machine learning y deep learning)

Y para el tema de clasificación de personas con el ojo, se ha encontrado.

Por otro lado, en temas de fine tunning con redes neuronales, se tiene