\capitulo{6}{Trabajos relacionados} \label{capitulo6}

La utilización del iris como elemento biométrico que permite el acceso a los dispositivos electrónicos es un tema de estudio candente en los últimos años, 
tal como se explica en la \nameref{capitulo1}.

A continuación se describen algunos de los trabajos centrados en la identificación de individuos a través de su iris en los que se ha basado este proyecto.

\section{Sistema clasificador de iris}

En este trabajo final de grado\cite{tfg_iris_2020} de la Universidad de Burgos, se desarrolla un sistema de clasificación basado en el iris. Para ello, se utiliza el \textit{dataset} 
de CASIA-V1. Este se somete primero a segmentación, utilizando el método de Wildes, y luego a normalización. 

Posteriormente, se extraen los atributos más identificativos de cada iris,
 utilizando una red que ya ha sido pre-entrenada para dicho fin\cite{lozej_end--end_2018}. Finalmente, se procede a la clasificación de individuos utilizando técnicas de \textit{machine learning}.

Como se ha comentado anteriormente, el código para la carga de imágenes y los procesos de segmentación y normalización del iris desarrollados en este proyecto, han sido utilizados de este trabajo como 
base de las mismas secciones en el \textit{pipeline}. La \textit{pipeline} se explica más detalladamente en el Anexo \ref{anx:pipeline}.

\section{Iris Recognition Development Techniques: A Comprehensive Review}

En este artículo \cite{malgheet_iris_2021}, los investigadores de la Universiti Putra Malaysia hablan de siete pasos en los que se divide un sistema de reconocimiento del iris:
\begin{enumerate}
\item adquisición
\item preprocesamiento
\item segmentación
\item normalización
\item extracción de características
\item selección de features únicos y característicos
\item clasificación. 
\end{enumerate}
 
 Este \textit{paper} también describe una falta de trabajos entorno a \textit{datasets} de baja calidad y realza que los sistemas de reconocimiento del iris se vuelven 
 poco efectivos cuando las imágenes tienen rotaciones or reflejos, algo que intentamos de aplicar a nuestro proyecto, con el \textit{data augmentation} (sección \ref{dataaugmentation}).

 El mismo articulo resume los distintos \textit{dataset} utilizados para estos estudios de reconocimiento de iris en los últimos años, el tipo de ruido utilizado en cada uno de ellos,
 así como describe a grandes rasgos cual es el método seguido en ellos para realizar el reconocimiento de inidivíduos a través de sus imágenes oculares.
 
 Los mas comunes son los tipos de segmentación tradicional, basado en la creación de círculos estandarizados, y aunque aparece un incremento del uso de las redes neuronales para la segmentación en los últimos años.
 También se describen distintas técnicas de normalización del iris y extracción de sus características, así como distintos enfoques en el cálculo del \textit{accuracy}.

\section{Iris Recognition Using Wavelet Transform and Artificial Neural Networks}

En este artículo de Hadeel N Abdullah \cite{abdullah_iris_2015}, perteneciente a la University of Technology de Iraq, se lleva a cabo un pre-procesamiento de un \textit{dataset} de imágenes oculares que incluye la extracción 
del iris utilizando la transfomrada de \textit{Hough}\footnote{Tiene como premisa que los bordes del ojo se pueden considerar círculos y se extrae el iris utilizando dicha forma.} y la normalización con \textit{Daugman}, ya explicado en la sección \ref{subsubsec:normalizacion}. 

Posteriormente, tras eliminar el ruido de la imagen, la extracción se realiza con transformaciones de Wavelet\footnote{Transforma los datos de las imágenes con la transformada de Wavelet.}. Finalmente, se crea una red neuronal, donde se utiliza el \textit{mean-squared error} para calcular los pesos en la red.

\section{Reliable pupil detection and iris segmentation algorithm based on SPS}

En este \textit{paper} \cite{minaee_deepiris_2019} encontramos el desarrollo de técnicas de deep learning para el reconocimiento del iris basado en una \textit{convolutional neural network} residual. utilizando una red pre-entrenada 
de ResNet50\footnote{Red neuronal con 50 capas, entrenada para reconocer imágenes de entre mil categorías.} y técnicas de \textit{fine-tuning} con el \textit{dataset} IIT Delhi\footnote{Accesible desde \url{https://www4.comp.polyu.edu.hk/~csajaykr/IITD/Database_Iris.htm}}. 

En este artículo, no se utiliza \textit{data augmentation} ni se pre-procesan las imágenes antes de aplicar el \textit{fine-tuning}. La red neuronal es entrenada con \textit{cross-entropy loss function}. 

\section{ An experimentalstudy of deep convolutional features for iris recognition}

En este artículo \cite{minaee_experimental_2017}, se hizo uso del \textit{dataset} CASIA 10k\footnote{Accesible desde \url{http://www.nlpr.ia.ac.cn/pal/CASIA10K.html}} y la arquitectura VGG16, para realizar una \textit{PCA}\footnote{\textit{Principal Components Analisis} es un método de reducción de dimensiones.} y extraer los elementos
más característicos de las imágenes. En una segunda fase, los investigadores utilizan algoritmos de \textit{machine learning} para clasificar las imágenes, consiguiendo unos altos porcentajes de acierto.

\section{Deep Learning-Based Feature Extraction in Iris Recognition: Use Existing Models, Fine-tune or Train From Scratch?}


En este artículo\cite{boyd_deep_2020}, los investigadores de la Universidad de Notre Dame se preguntan si vale la pena, teneniendo un \textit{dataset} de imágenes oculares de un tamaño considerable,
entrenar a una red neuronal desde cero o aplicar técnicas de \textit{fine-tuning} sobre una red ya pre-entrenada.

Para contestar a esta pregunta, utilizan técnicas de \textit{deep learning} para clasificar imágenes de ojos de el \textit{dataset} CASIA 10k sin aplicar técnicas de \textit{data augmentation} sobre el. 
Para la segmentation y normalización del iris se utiliza OSIRIS\footnote{Herramienta de código abierto utilizada para el reconocimiento del iris \url{https://github.com/tohki/iris-osiris}.}.

Para la parte de \textit{fine-tuning}, se utiliza la red pre-entrenada ResNet50 y posteriormente, se realiza la clasificación con SVM\footnote{Support vector machine es un método de aprendizaje supervisado.}.
La conclusión que se extrae del artículo es que, al menos en este caso, el clasificador funciona mejor al adaptar una red neuronal con \textit{fine-tuning} antes que crear la nuestra propia, aún siendo el 
\textit{dataset} lo suficientemente grande para hacerlo.
