\apendice{Plan de Proyecto Software}

\section{Introducción}

La planificación del proyecto ha sido una parte esencial, puesto que ha permitido la coordinación entre estudiante y tutores de forma online.

\section{Planificación temporal}

Para la planificación temporal se ha utilizado la metodología SCRUM basada en sprints, generalmente de una o dos semanas.

\subsection{Sprint 1}

\begin{itemize}
\item Creación de un repositorio para el control de las versiones y el seguimiento del trabajo.
\item Investigación sobre la materia de estudio (antecedentes, casos de uso, estado del arte)
\item Investigación sobre posibles dataset de iris para el estudio.
\item Investigación sobre la metodogía que mejor se adapte al proyecto.
\item Estudio sobre el uso de data augmentation sobre datasets de iris.
\item Investigación sobre el uso de pipelines para la mejora de la reproducibilidad del trabajo.
\item Investigación sobre los transformes de scikit-learn
\end{itemize}

\subsection{Sprint 2}

\begin{itemize}
\item Investigación sobre pipelines que se ejecuten de forma condicional.
\item Investigación sobre posibles redes neuronales para la segmentación del iris.
\item Investigación para el uso de data augmentation en el paquete Keras.
\end{itemize}

\subsection{Sprint 3}

\begin{itemize}
\item Clarificación de objetivos del proyecto.
\item Pruebas con la utilización de ruido gaussiano con standard deviation aleatorio
\item Investigación sobre las transformaciones afines como método para el data augmentation.
\item Implementación del pipeline.
\end{itemize}

\subsection{Sprint 4}

\begin{itemize}
\item Pruebas con las transformaciones afines de Keras.
\end{itemize}

\subsection{Sprint 5}

\begin{itemize}
\item Pruebas con el transformador de Keras.
\end{itemize}

\subsection{Sprint 6 y 7}

\begin{itemize}
\item Aplicaciñon de fine tunning con redes neuronales.
\end{itemize}

\subsection{Sprint 8}

\begin{itemize}
\item Reconfigurar el pipeline para que el data augmentation solo se aplique al conjunto de entreamiento.
\item Prevención de imagenes no segmentadas correctamente para prevenir que entren en el proceso de normalización.
\item Incorporación del modelo de deep learning al pipeline.
\item Mejora de la configuración del pipeline.
\end{itemize}

\subsection{Sprint 9}

\begin{itemize}
\item Creación de los modelos de clasificación.
\end{itemize}

\section{Estudio de viabilidad}

En este apartado se redacta la viabilidad del proyecto, tanto económica como legal, que potencialmente habría costado el proyecto.

\subsection{Viabilidad económica}

En primer lugar, se encuentran los cálculos económicos del proyecto, que se pueden dividir en gastos de \textit{software}, \textit{hardware} y de personal.

\subsubsection{\textit{Software}}

Los recursos utilizados para la realización gratuitos y de código abierto, por lo que no se prevé ningún gasto en este apartado.

\subsubsection{\textit{Hardware}}

Para la realización del proyecto, se ha contado con una máquina de la Universidad de Burgos, cuyas características se reflejan en la tabla\ref{caracteristicaspc}.
En base a las características, el precio para esta máquina se establece entre los 2.500 y los 3.000 euros, aunque se ha de tener en cuenta que la amortización del mismo va más allá de la realización de este proyecto.

\subsubsection{Personal}

Finalmente, se ha de tener en cuenta que el proyecto se ha llevado a cabo por el estudiante y dos tutores.

Considerando que el proyecto se ha llevado a cabo en aproximadamente en 6 meses, se considerará que el estudiante trabajaría a tiempo completo durante ese tiempo,
con un sueldo de 1500 euros al mes, cuyo gasto total para la universidad sería de unos 2400 euros al mes. Tabla \ref{tabla:gastosestudiante}

\tablaSmall{Gastos de contratación del estudiante.}{l c}{gastosestudiante}
{ {Gasto} & Cantidad en euros\\}{ 
Base de cotización & 1500  \\
Coste seguridad social empresa & 898\\
Total universidad & 2398\\
} 


Por su parte, los dos profesores estarían contratados por ese mismo periodo de tiempo, pero trabajando a media jornada., con un salario de 1800 euros al mes por la media jornada, cuyo gasto para la unviersidad sería de unos 2753 euros al mes por cada tutor. Tabla \ref{tabla:gastostutores}

\tablaSmall{Gastos de contratación de los tutores (por tutor).}{l c}{gastostutores}
{ {Gasto} & Cantidad en euros\\}{ 
Base de cotización & 1800  \\
Coste seguridad social empresa & 953\\
Total universidad & 2753\\
} 
\subsection{Gasto total}

Por lo tanto, el gasto total se establece en, como se desglosa en la siguiente tabla \ref{tabla:gastostotales}

\tablaSmall{Gastos totales del proyecto.}{l c}{gastostotales}
{ {Gasto} & Cantidad en euros\\}{ 
\textit{Hardware} & 2500 \\
Contratación estudiante & 14400  \\
Contratación tutores & 33036\\
Total & 49936\\
} 

\subsection{Viabilidad legal}

Como se ha comentado anteriormente, el \textit{software} utilizado es gratuito y libre, y está sujeto a las licencias recogidas en la tabla \ref{tabla:licencialibrerias}.

\tablaSmall{Licencias de las librerías utilizadas.}{l c c c}{licencialibrerias}
{ {Librería} & Versión & Licencia\\}{ 
Anaconda & 2.1.1 & Libre\\
Imageio & 2.19.3 & BSD\\
Jupyter Notebook & 4.11 & BSD\\
Keras & 2.7 & MIT\\
Matplotlib & 3.1 & BSD\\
Numpy & 1.19.5 & BSD\\
OpenCV & 4.6 & BSD\\
Pandas & 1.4.3 & BSD\\
Python & 3.10.5 & PFS \\
Scikit-Learn & 1.0 & BSD\\
TensorFlow & 2.9.1 & Apache\\
} 

El \textit{dataset} utilizado tiene fines de investigación, pero no se permite la distribución alterada ni la misma y su uso no referenciado.

Finalmente, se han utilizado los iconos de \url{https://www.flaticon.com/} para la elaboración de las gráficas. La licencia de los mismos no permiten comercializarlos
 o modificarlos.
