\capitulo{4}{Técnicas y herramientas}

Se ha utilizado el dataset de CASIA para la imágenes de ojos.

Así mismo, se ha utilizado la red neuronal preentrenada Basada en U-Net y accesible desde https://github.com/jus390/U-net-Iris-segmentation, la cual ya había sido
entrenada para la segmentación del iris.

Finalmente, para el último proceso del proyecto, se utiliza imagenet como red neuronal preentrenada central.

Además se ha utilizado python para todo el proyecto, esto quiere decir preprocesado, creación y utilización de redes neuronales y clasificación contécnicas de 
machine learning.

Entre las principales librerías utilizadas se encuentran:

* os, para el acceso a los directorios
* numpy, para trabajar con las imágenes a nivel de arrays
* scikit-image, para la transformación de las imágenes y el uso de dataset
* tensorflow, para la modificación de las redes neuronales
* keras, para el manejo de las redes neuronales
* matplot, para las gráficas


