\capitulo{5}{Aspectos relevantes del desarrollo del proyecto}

El proyecto puede dividirse en una fase previa y tres propuestas diferentes para la identificación biomédica de personas a través de una imagen de su ojo.

La primera fase trata de la optimización del código previo y la creación de un pipeline con una configuración que permitiese un mejor control de las funciones. 
También del  data augnmentation, en el cual se aplica tanto ruido gausseano (de 2.5, 5 y 7.5) como transformaciones afines (), siendo realizadas de 
forma aleatoria, con lo cual los supuestos pueden ser a) imagen sin data augnmentation, b)imagen con ruido gausseano, c)imagen con transformaciones afines y 
d)imagen con ruido gausseano y transformaciones afines. Este dataset será el base para todo el proyecto.

La segunda fase se trata de la elección de la mejor forma de clasificar las imagenes de entrada:

1.La propuesta en el TFG anterior, en la cual se realiza un preprocesamiento de las imágenes del dataset original. Este preprocesamiento consiste primer lugar 
en la segmentación de las imágenes del iris con una red neuronal preentrenada precisamente para realizar esta acción. En segundo lugar, se realiza una extracción
del iris a través de una binarización de las partes del ojo y una extracción del iris, a la que se le aplica una normalización para que quede proyectado.
La siguiente fase de esta primera propuesta es la extracción de características (quitandole las dos últimas capas a una red neuronal) de la imagen normalizada
 para posteriormente utilizar una red neuronal preentrenada con imagenet (de hecho 3 redes de la cual se elige la mejor). Posteriormente, estas características
 extraidas se pasan a modelos de ML, que son los que realizarán la clasificación. 

 Todo el proyecto queda establecido en una pipeline
