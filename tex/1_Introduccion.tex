\capitulo{1}{Introducción}

El iris se utiliza como elemento de reconocimiento biométrico, tanto por su inmutabilidad a lo largo del tiempo como por resultar un valor único y 
personal, que supone que dos personas no tendrían el mismo iris (05 Iris Recognition Developmen Techniques: A Comprehensive Review).

En este estudio, se compararán técnicas de redes neuronales completas como una combinación de redes neuronales para la segmentación del ojo, la extracción
de características y finalmente, con técnicas de Machine Learning.

Para ello, vamos a utilizar una red neuronal preentrenada de ImageNet a la que le aplicaremos fine-tuning y compararemos los resultados con la combinación de
una red neuronal preentrenada para la segmentación, un preprocesamiento para extraer y normalizar el iris, la red preentrenada ImageNet para extraer las características principales
(sin realizar fine-tuning) y machine learning para el cluster (identificación del individuo).

\section{Outline}

El resto del docuemnto se estructura de la siguiente manera. El capítulo 2 contiene los objetivos del proyecto. El capítulo 3 contiene los conceptos teóricos necesarios
para entender el proyecto. El capítulo 4 muestra las técnicas y herramientas utilizadas en el desarrollo de este trabajo. En el capítulo 5 se muestran los Aspectos
más relevantes que se han desarrollado. En el capítulo 6 los trabajos relacionados y en el capítulo 7 las conclusiones y las líneas de trabajo futuras.
