\capitulo{1}{Introducción}

El iris se utiliza como elemento de reconocimiento biométrico, tanto por su inmutabilidad a lo largo del tiempo como por resultar un valor único y 
personal, que supone que dos personas no tendrían el mismo iris (05 Iris Recognition Developmen Techniques: A Comprehensive Review).

En este estudio, se compararán técnicas de redes neuronales completas como una combinación de redes neuronales para la segmentación del ojo, la extracción
de características y finalmente, con técnicas de Machine Learning.

Para ello, vamos a utilizar una red neuronal preentrenada de ImageNet a la que le aplicaremos fine-tuning y compararemos los resultados con la combinación de
una red neuronal preentrenada para la segmentación, un preprocesamiento para extraer y normalizar el iris, la red preentrenada ImageNet para extraer las características principales
(sin realizar fine-tuning) y machine learning para el cluster (identificación del individuo).
